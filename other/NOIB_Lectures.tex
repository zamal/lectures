% Created 2011-01-05 Ср. 23:16
\documentclass[12pt, russian, oneside, article]{ncc}
\usepackage[utf8]{inputenc}
\usepackage[T1]{fontenc}
\usepackage{fixltx2e}
\usepackage{graphicx}
\usepackage{longtable}
\usepackage{float}
\usepackage{wrapfig}
\usepackage{soul}
\usepackage{textcomp}
\usepackage{marvosym}
\usepackage{wasysym}
\usepackage{latexsym}
\usepackage{amssymb}
\usepackage{hyperref}
\tolerance=1000
\usepackage[math]{pscyr}
\usepackage{indentfirst}
\providecommand{\alert}[1]{\textbf{#1}}
\begin{document}



\title{Нормативное обеспечение информационной безопасности}
\author{Максим Захаров}
\date{05 Январь 2011}
\maketitle

\setcounter{tocdepth}{3}
\tableofcontents
\vspace*{1cm}

\href{file:///home/maxim/Documents/Git/lectures/other/NOIB_Lectures.pdf}{Скачать в PDF}

\section{ОБЕСПЕЧЕНИЕ БЕЗОПАСНОСТИ СЕТЕЙ ЭЛЕКТРОСВЯЗИ}
\label{sec-1}
\subsection{Обеспечение безопасности сетей электросвязи}
\label{sec-1_1}
\subsubsection{Общие положения и терминология}
\label{sec-1_1_1}


Основными функциями сетей электросвязи (СЭС), которые являются составными компонентами сети связи общего пользования единой сети электросвязи России, являются приём, обработка, хранение, передача и предоставление требуемой информации пользователям и органам государственного управления для её последующего применения.

СЭС предназначены для оказания услуг связи любому пользователю путём предоставления открытых информационных ресурсов и информации, не содержащей ССГТ или информации, доступ к которой ограничен в соответствии с законодательством РФ.

Под безопасностью СЭС понимается её способность противодействовать определённому множеству угроз, преднамеренных или непреднамеренных дестабилизирующих воздействий на входящие в состав сетей средства, линии связи и технологические процессы (протоколы), что может привести к ухудшению качества услуг, предоставляемых СЭС.

Дестабилизирующими воздействиями являются действия, источником которых являются физические и технологические процессы внутреннего или внешнего по отношению к СЭС характера, приводящие к выходу из строя элементов сети.

Под инфокоммуникационной структурой СЭС понимается совокупность информационных ресурсов и инфраструктуры СЭС.

Инфраструктура СЭС определяется как совокупность средств связи, линий связи, сооружений связи, технологических систем связи, технологий и организационных структур, обеспечивающих информационное взаимодействие компонентов СЭС.

Информационные ресурсы СЭС --- это совокупность хранимых (используемых для обеспечения функционирования процессов СЭС), обрабатываемых и передаваемых данных, содержащих информацию пользователей и/или системы управления СЭС.

Под устойчивостью функционирования СЭС понимается способность СЭС выполнять свои функции при выходе из строя части элементов сети в результате дестабилизирующих воздействий.

Меры обеспечения безопасности включают в себя набор функций, определяющих возможности механизмов обеспечения безопасности СЭС по непосредственной или косвенной реализации требований к безопасности.

Механизмом обеспечения безопасности СЭС является взаимоувязанная совокупность организационных, аппаратных, программных и программно-аппаратных средств, способов, методов, правил и процедур, используемых для реализаций требований к безопасности СЭС.

Нарушитель безопасности СЭС --- физическое или юридическое лицо, преступная группа, процесс или событие, производящее преднамеренные или непреднамеренные воздействия на инфокоммуникационную структуру СЭС, приводящие к нежелательным последствиям для интересов пользователей услугами связи, операторов связи и/или органов государственного управления.

Под риском нарушения безопасности СЭС понимается вероятность причинения ущерба СЭС или её компонентам вследствие того, что определённая угроза реализуется в результате наличия определённой уязвимости в СЭС.

Угрозой безопасности СЭС является совокупность условий и факторов, создающих потенциальную или реально существующую опасность нанесения ущерба СЭС или её компонентам.

Уязвимость СЭС определяется как недостаток или слабое место в средстве связи, техническом процессе (протоколе) обработки/передачи информации, мероприятиях и механизмах обеспечения безопасности СЭС, позволяющие нарушителю совершать действия, приводящие к успешной реализации угроз безопасности.

Система обеспечения безопасности системы связи общего пользования --- совокупность служб безопасности операторов средств ССОБ и используемых ими механизмов обеспечения безопасности, взаимодействующая с органами управления СЭС, организация и функционирование которой осуществляется по нормам правилами и обязательным требованиям, установленных в области связи.

Под службой безопасности СЭС понимается организационная техническая структура оператора СЭС, реализующая политику безопасности оператора связи и обеспечивающая функционирование системы обеспечения безопасности СОБ.

Политикой безопасности оператора связи является совокупность документированных правил, процедур, практических приёмов и руководящих принципов в области обеспечения безопасности, которыми должен руководствоваться оператор связи.

Сеть связи --- технологическая система, включающая в себя средства и линии связи и предназначенная для электросвязи или почтовой связи.

Под электросвязью понимаются любые излучения, передача или приём знаков, сигналов, голосовой информации, письменного текста, изображений, звуков или сообщений любого рода по радиосистеме проводной, оптической или другим электромагнитным системам.
\subsubsection{СЭС, их информационные ресурсы и проблемы обеспечения безопасности}
\label{sec-1_1_2}


Сети электросвязи являются средой переноса сообщений любого рода в виде электрических сигналов. Сообщения содержат информацию пользователя, которая может быть открытой, закодированной, зашифрованной или скремблированной (что для сети электросвязи является неопределяющим), и служебную информацию (например адрес получателя). Сеть электросвязи должна обеспечить целостность передаваемых сообщений и своевременность их доставки адресату.

Открытость сетей электросвязи не должна означать полную доступность ко всем ее информационным ресурсам и отсутствие контроля их использования. В сети электросвязи должна быть обеспечена защита собственной, служебной информации, предназначенной для управления работой сети или служб сети.

К информационным ресурсам сетей электросвязи, требующим защиты со стороны оператора связи, могут быть отнесены:
\begin{itemize}
\item сведения об абонентах, базы данных;
\item информация управления;
\item данные, содержащие информацию пользователей (обеспечение доступности и целостности);
\item программное обеспечение систем управления сетями электросвязи;
\item сведения о прохождении, параметрах, загрузке (использовании) линий связи магистральных сетей;
\item обобщенные сведения о местах дислокации узлов связи и установленном сетевом оборудовании;
\item сведения, раскрывающие структуру используемых механизмов обеспечения безопасности сети электросвязи.
\end{itemize}

Необходимость рассмотрения проблем обеспечения безопасности сетей электросвязи обусловлена:
\begin{itemize}
\item динамикой развития сетей электросвязи и их интеграцией с глобальными сетями связи, в том числе с Интернет;
\item совершенствованием применяемых ИТ;
\item ростом числа пользователей услугами связи и расширением спектра предоставления услуг связи;
\item увеличением объемов хранимой и передаваемой информации;
\item территориальной рассредоточенностью сложных информационно-телекоммуникационных структур;
\item недостаточностью в сетях электросвязи необходимых механизмов обеспечения безопасности.
\end{itemize}

Эти проблемы существенно повышают уязвимость сетей, способствуют появлению новых угроз безопасности и определяют необходимость комплексного решения задач по обеспечению безопасности сетей электросвязи путем:
\begin{itemize}
\item организации эффективного, безопасного управления и взаимодействия сетей;
\item поддержания гарантированных качественных характеристик процессов обработки информации в сетях электросвязи (качества обслуживания) в условиях возможных ВН на инфокоммуникационную структуру сетей электросвязи;
\item создания в сетях электросвязи надежных и защищенных каналов по пропуску определенных категорий трафика, из совокупности которого могут быть извлечены сведения, способные нанести ущерб безопасности Российской Федерации;
\item противодействия проявлению терроризма на сетях электросвязи, в том числе экстремистским действиям.
\end{itemize}

Решение данных проблем является функцией СОБ сетей электросвязи ССОП и служб безопасности операторов связи в рамках общих положений по безопасности сетей электросвязи, предлагаемых настоящим стандартом.
\subsubsection{Основными целями обеспечения безопасности СЭС}
\label{sec-1_1_3}


Основными целями обеспечения безопасности сетей электросвязи являются:
\begin{itemize}
\item достижение устойчивого функционирования и успешного выполнения заданных функций сетью электросвязи, в условиях возможного ВН, способного привести  к нарушению конфиденциальности, целостности, доступности или подотчетности;
\item обеспечение доступности услуг связи, особенно услуг экстренного обслуживания в чрезвычайных
\end{itemize}
ситуациях, в том числе и в случае террористических актов.

Основными задачами обеспечения безопасности сетей электросвязи являются:
\begin{itemize}
\item своевременное выявление, оценка и прогнозирование источников угроз безопасности, причин и условий, способствующих нанесению ущерба, нарушению нормального функционирования и развития сетей электросвязи на всех уровнях иерархии единой сети электросвязи России (международном, междугороднем, зоновом, местном, на уровне пользования услугами связи и т.д.);
\item выявление и устранение уязвимостей в средствах связи и сетях электросвязи;
\item предотвращение, обнаружение угроз безопасности, пресечение их реализации и своевременная ликвидация последствий возможных ВН, в том числе и террористических действий;
\item организация системы пропуска приоритетного трафика по сети электросвязи в случае чрезвычайных ситуаций, организация бесперебойной работы международной аварийной службы;
\item совершенствование и стандартизация применяемых мер обеспечения безопасности сетей электросвязи.
\end{itemize}

Операторами связи могут быть определены дополнительные цели и задачи обеспечения безопасности сетей электросвязи в зависимости от выполняемых организацией связи функций и ее бизнес-целей, но формулировка целей и задач должна быть независима от способов их реализации.

Оператор связи при осуществлении процесса управления функционированием сети электросвязи должен минимизировать возможные негативные ВН для обеспечения выполнения основных целей организации связи, в том числе и бизнес-процессов. Это достигается путем интегрирования в систему управления функционированием сети электросвязи процесса управления рисками. На каждой стадии жизненного цикла сетей электросвязи (проектирование, строительство, реконструкция, развитие и эксплуатация) должна осуществляться деятельность по поддержанию управления рисками, основой которой являются процессы идентификации и оценки рисков. 

Оценка риска при обеспечении безопасности сетей электросвязи должна производиться на основе анализа уязвимостей сетей электросвязи и угроз, способных реализовать эти уязвимости.

Угрозы могут способствовать причинению ущерба пользователям услугами связи, операторам и/или органам государственного управления.

За основу классификации угроз безопасности сетей электросвязи рекомендуется классификацию, установленную ГОСТ Р 51275, в соответствии с которой угрозы могут быть классифицированы:
\begin{itemize}
\item по природе возникновения: объективные (естественные) или субъективные (искусственные);
\item по источнику возникновения: внешние или внутренние.
\end{itemize}
\subsubsection{Угрозы безопасности СЭС. Модели угроз.}
\label{sec-1_1_4}


Источником угроз безопасности СЭС могут быть:
\begin{enumerate}
\item Субъект.
\item Материальный объект.
\item Физическое явление.
\end{enumerate}

В процессе обеспечения безопасности СЭС необходимо выявление всех возможных угроз в инфокоммуникационной сети.

Полное множество угроз безопасности не поддаётся формализации. Это связано с тем, что архитектура современных СЭС, используемые технологии обработки, передачи, хранения информации подвержены большому количеству субъективных дестабилизирующих воздействий. Но чем больше будет выявлено возможных угроз безопасности, тем точнее будет оценено состояние безопасности СЭС.

К основным возможным угрозам безопасности СЭС могут быть отнесены следующие угрозы:
\begin{enumerate}
\item Уничтожение информации и/или других ресурсов.
\item Искажение или модификация информации.
\item Мошенничество.
\item Кража, утечка, потеря информации или других ресурсов.
\item Несанкционированный доступ.
\item Отказ в обслуживании.
\end{enumerate}

Каждая выявленная угроза в соответствии с выбранной методикой оценкой риска должна ранжироваться по вероятности своего возникновения для последующего анализа рисков и оценки величины возможного ущерба СЭС от реализации угроз.

Пример трёхуровневой градации вероятности возникновения угроз.

Описание показателей вероятности возникновения угроз.


\begin{center}
\begin{tabular}{ll}
 Показатель вероятности  &  Описание действий нарушителя                            \\
\hline
 Маловероятный           &  Нарушитель обладает очень незначительными техническими  \\
                         &  возможностями для реализации угрозы или мотивация для   \\
                         &  нарушителя очень низкая.                                \\
\hline
 Вероятна                &  Технические возможности, необходимые для реализации     \\
                         &  угрозы не слишком высоки и разрешимы без большого       \\
                         &  усилия, кроме того должно быть разумное для нарушителя  \\
                         &  побуждения, чтобы реализовать угрозу.                   \\
\hline
 Возможна                &  На СЭС отсутствуют механизмы обеспечения безопасности,  \\
                         &  используемые для противодействия этой угрозе и          \\
                         &  побуждение для нарушителя весьма высока.                \\
\end{tabular}
\end{center}



В целях учёта всех возможных сфер проявления угроз для каждой конкретной СЭС необходимо разрабатывать модель угроз безопасности.

Модель угроз безопасности СЭС представляет собой нормативный документ, которым должен руководствоваться заказчик при задании требований безопасности к сети и разработчик, создающий эту сеть и службы обеспечения ИБ сети при её эксплуатации.

Модель угроз должна включать:
\begin{enumerate}
\item Описание ресурсов инфокоммуникационной структуры (объектов безопасности) СЭС, требующих защиты.
\item Описание источников формирования дестабилизирующих воздействий и их потенициальных возможностей.
\item Стадии жизни цикла СЭС, в т. ч. определяющий её технологический и эксплуатационный этапы.
\item Описание процесса возникновения угроз и путей их практической реализации.
\end{enumerate}

К качестве приложения модель угроз безопасности должна содержать полный перечень угроз и базу данных о выявленных нарушениях безопасности СЭС с описанием обстоятельств, связанных с обнаружением нарушений.

В соответствии с разработанной моделью угроз оценивается опасность угроз для каждой группы идентифицированных ресурсов инфокоммуникационной структуры СЭС и услуг связи и определяются возможная мера обеспечения безопасности для противодействия каждой конкретной угрозе.
\subsection{Нарушители безопасности СЭС}
\label{sec-1_2}
\subsubsection{Модель нарушителя}
\label{sec-1_2_1}


Угрозы безопасности СЭС реализуются нарушителями безопасности через выявленные уязвимости инфокоммуникационной структуры сети, в которую они могут быть внесены на технологическом и/или эксплуатационном этапах её жизненного цикла.

Угрозы безопасности могут изменяться. Уязвимость может существовать на протяжении всего срока эксплуатации СЭС или конкретного протокола, если она своевременно не устраняется разработчиком или по его представлению службами эксплуатации оператора связи.

Нарушителями безопасности СЭС могут быть:
\begin{enumerate}
\item Террористы и террористические организации.
\item Конкурирующие организации и структуры.
\item Спецслужбы иностранных государств и блоков государств.
\item Криминальные структуры.
\item Взломщики программных продуктов ИТ, использующихся с системах связи.
\item Бывшие сотрудники организации связи.
\item Недобросовестные сотрудники и партнёры.
\item Пользователя услугами связи и др.
\end{enumerate}

Основными мотивами нарушений безопасности СЭС могут быть:
\begin{enumerate}
\item Месть.
\item Достижение денежной выгоды.
\item Хулиганство и любопытство.
\item Профессиональное самоутверждение.
\end{enumerate}

Для учёта всех возможных воздействий нарушителя и определения его категории разрабатывается модель нарушителя безопасности СЭС, под которой понимается абстрактная (формализованное или неформализованное) описание нарушителя безопасности.

Задача построения модели нарушителя безопасности СЭС состоит в определении:
\begin{enumerate}
\item Штатных объектов или элементов сети, к которым возможен доступ.
\item Субъектов, допущенных к работе с оборудованием сети в период её проектирования, разработки, развёртывания и эксплуатации.
\item Перечня соответствия объета доступа к субъекта, которые могут быть потенциальными нарушителями.
\end{enumerate}

При определении потенциального нарушителя и составления его модели необходимо исходить из того, что нарушитель может быть как законным абонентом сети (принадлежать к персоналу, непосредственно работающему с абонентскими терминалами), так и посторонним лицом, пытающимся непосредственно или с помощью имеющихся у него технических и программных средств получить доступ к информационным ресурсам и инфраструктуре сети.
\subsubsection{Направленность и характер воздействий нарушителя безопасности СЭС}
\label{sec-1_2_2}


Воздействия нарушителя в основном направлены на ухудшение качественных характеристик СЭС и могут осуществляться как правило путём поиска и использования эксплуатационных и технологических уязвимостей.

Воздействия нарушителя могут осуществляться:
\begin{enumerate}
\item По каналам абонентского доступа, в т. ч. и беспроводным.
\item По внутренним линиям связи.
\item С рабочих мест систем управления и технического обслуживания.
\item По недекларированным каналам доступа.
\end{enumerate}

При этом могут использоваться как штатные, так и специальные средства связи.

Воздействия нарушителя могут носить как непреднамеренный (случайный), так и преднамеренный характер.

Непреднамеренные случайные воздействия могут быть спровоцированы:
\begin{enumerate}
\item Недостаточной надёжностью средств связи.
\item Ошибками обслуживающего персонала.
\item Природными явлениями.
\item Другими объективными дестабилизирующими факторами.
\end{enumerate}

Преднамеренные воздействия могут быть:
\begin{enumerate}
\item Активными.
\item Пассивными.
\item Не преследующими цели.
\end{enumerate}

Активные действия нарушителя предусматривают вмешательство в работу СЭС, нарушение режимов её функционирования и снижение качества обслуживания вплоть до полного прекращения предоставления услуг связи пользователям.

Основные цели активных действий:
\begin{enumerate}
\item Подрыв репутации оператора-конкурента путём нарушения доступности услуг связи и (или) ухудшения её характеристик.
\item Несанкционированное использование услуг.
\end{enumerate}

Пассивные действия нарушителя предполагают нанесение вреда абоненту (пользователю услугами связи) путём использования выявленных уязвимостей СЭС, но не наносящие прямого вреда СЭС. Целью таких действий могут являться:
\begin{enumerate}
\item Перехват персональных данных пользователей (например паролей для регистрации терминалов).
\item Перехват данных о финансовых сделках с целью нанесения ущерба бизнесу.
\item Наблюдение за выполняемым процессом (подготовка для новых атак, активных действий).
\item Поиск идеологических, политических выгод.
\item Шантаж, вымогательство.
\end{enumerate}

Действия, непреследующие цели (хулиганство) --- действия, не ставящие цели нанесения вреда конкретному физическому объекту или лицу.
\subsection{Критерии безопасности СЭС. Последствия нарушений безопасности СЭС.}
\label{sec-1_3}


Критерии безопасности СЭС:
\begin{enumerate}
\item Конфиденциальность инфокоммуникационной структуры СЭС.
\item Целостностью информации услуг связи.
\item Доступностью информации услуг связи.
\item Подотчётностью действий в сети.
\end{enumerate}

Под конфиденциальностью инфокоммуникационной структуры СЭС понимают свойства, позволяющие ограничить НСД к инфокоммуникационной структуре СЭС и (или) не раскрывать содержания информации лицам, объектам или процессам. Нарушение конфиденциальности --- несанкционированное раскрытие информации управления, персональных данных пользователей.

Под целостностью информации услуг связи понимают состояние СЭС, при котором обеспечивается неизменность информации и доступность услуг связи для пользователей независимо от преднамеренного или случайного несанкционированного воздействия нарушителя на инфокоммуникационную структуру сети в т. ч. в чрезвычайных ситуациях.

Нарушение целостности --- несанкционированная модификация, разрушение информационных ресурсов и структуры СЭС.

Под доступностью информации услуг понимается способность СЭС обеспечить пользователям согласованные условия доступа к предоставляемым услугам связи и их получение в т. ч. в условиях возможных воздействиях нарушителя на инфокоммуникационную структуру СЭС.

Нарушение доступности --- нарушение доступа к пользованию информацией и услуг связи.

Под подотчётностью понимают свойство, которое обеспечивает однозначное отслеживание действий в сети любого объекта.

Нарушение подотчётности --- отрицание действий в сети (например участие в совершённом сеансе связи) или подделка (создание информации) и претензии, которые якобы были получены от другого объекта или посланы другому объекту

В таблице показана взаимосвязь основных угроз и критериев безопасности СЭС.



\begin{center}
\begin{tabular}{llll}
 Вид угрозы              &  К  &  Ц  &  Д  \\
\hline
 Уничтожение информации  &  -  &  +  &  +  \\
 и (или) др. ресурсов    &     &     &     \\
\hline
 Искажение или           &     &     &     \\
 модификация информации  &  -  &  +  &  -  \\
\hline
 Мошенничество           &  +  &  +  &  +  \\
\hline
 Кража, утечка, потеря   &     &     &     \\
 утечка информации       &  +  &  +  &  +  \\
\hline
 НСД                     &  +  &  +  &  +  \\
\hline
 Отказ в обслуживании    &  -  &  +  &  +  \\
\end{tabular}
\end{center}



Нарушение конфиденциальности, целостности, доступности, подотчётности при потенциальном воздействии нарушителя может иметь следующие последствия для деятельности оператор связи и состояния инфокоммуникационной структуры СЭС:
\begin{enumerate}
\item Низкое потенциальное воздействие может привести к ограниченному неблагоприятному эффекту.
\item Умеренное потенциальное воздействие может привести к серьёзному неблагоприятному эффекту.
\item Высокое потенциальное воздействие может привести к тяжёлому или катастрофическому неблагоприятному эффекту.
\end{enumerate}

В соответствии с используемой оператором связи методикой оценки рисков и с учётом вероятностей возникновения угрозы и потенциального воздействия нарушителя по реализации данной угрозы должен определяться риск возможного нанесения ущерба СЭС.

Величина риска может классифицироваться 3 показателями, приведёнными в таблице. Описание показателей величины возможного риска. 


\begin{center}
\begin{tabular}{ll}
 Уровень значения показателя  &  Описание риска                                 \\
 величина риска               &                                                 \\
\hline
 Незначительный               &  Незначительные риски возникают, если атаки     \\
                              &  нарушителя являются маловероятными. Угрозы,    \\
                              &  причиняющие незначительные риски, считаются    \\
                              &  допустимыми                                    \\
\hline
 Существенные                 &  Существенные риски для соответствующих         \\
                              &  ресурсов представлены угрозами, которые,       \\
                              &  вероятно произойдут, даже если их              \\
                              &  воздействие является менее фатальным.          \\
                              &  Существенные риски должны быть минимизированы  \\
\hline
 Критический                  &  Критические риски возникают, когда             \\
                              &  появляется угроза ущерба интересам оператора   \\
                              &  сети и когда не требуется больших усилий       \\
                              &  потенциальному нарушителю, чтобы навредить     \\
                              &  этим интересам. Критические риски должны       \\
                              &  быть минимизированы с самым высоким            \\
                              &  приоритетом                                    \\
\end{tabular}
\end{center}
\subsection{Принципы обеспечения безопасности СЭС в условиях воздействия нарушителя}
\label{sec-1_4}


Обеспечение безопасности должно осуществляться с учётом основных принципов:
\begin{enumerate}
\item Комплексности использования всей совокупности нормативно-правовых актов, организационных и режимных мер, программных, аппаратных и программно-аппаратных методов защиты, обеспечивающих безопасное функционирование СЭС.
\item Защищённости сбалансированных интересов пользователей, операторов связи и органов государственного управления.
\item Управляемости методами, действиями и процедурами по обеспечению безопасности сетей электросвязи и контролю качества процессов передачи информации в условиях возможных ВН на инфокоммуникационную структуру сетей в соответствии с функциями системы управления сетью.
\item Непрерывности совершенствования методов, действий и процедур по обеспечению безопасности сетей электросвязи с учетом достигнутого отечественного и зарубежного опыта в условиях возможных ВН и изменения методов и средств этих воздействий.
\item Совместимости аппаратно-программных средств и технологий, применяемых в СОБ.
\end{enumerate}

\emph{Интересы пользователей} состоят в доверии к сети и предлагаемым услугам связи, в том числе доступности услуг (особенно экстренного обслуживания) в случае катастроф, включая террористические акты.

\emph{Интересы операторов связи} заключаются в выполнении ими своих обязательств перед пользователями услугами связи и защите от посягательств на свои финансовые и деловые интересы.

\emph{Интересы органов государственного управления} определяются необходимостью предъявления требований к безопасности сетей электросвязи, обеспечения соблюдения операторами связи предъявляемых им требований к безопасности, добросовестной конкуренции и защиты персональных данных пользователей.
\subsection{Общие требования к безопасности СЭС}
\label{sec-1_5}


На всех этапах проектирования, строительства, реконструкции, развития и эксплуатации сетей электросвязи и сооружений связи к ним должны предъявляться требования по обеспечению безопасного их функционирования, сопоставимые с возможными ВН на инфокоммуникационную структуру сетей электросвязи и ожидаемым ущербом от данных воздействий.

Требования к безопасности сетей электросвязи устанавливают федеральные органы исполнительной власти в области связи на основании законодательства в области связи и защиты информации, с учетом рекомендаций международных организаций по стандартизации, а также предложений отечественных саморегулируемых организаций в области электросвязи и лучшей практики отечественных операторов связи.

Требования по обеспечению безопасности конкретной сети электросвязи должны формироваться с учетом:
\begin{itemize}
\item целей, функций и задач решаемых оператором связи,
\item условий использования сети электросвязи в общей системе связи государства,
\item специфики используемой технологии передачи информации,
\item потенциальных угроз безопасности и возможных воздействий нарушителя,
\item реальных проектных и эксплуатационных ресурсов и существующих ограничений на функционирование сети электросвязи,
\item требований и условий взаимодействия с другими сетями электросвязи.
\end{itemize}

Предоставление и использование услуг и механизмов обеспечения безопасности может быть довольно дорогим относительно потерь при нарушении безопасности сетей электросвязи. Поэтому должно анализироваться соотношение между стоимостью мер по обеспечению безопасности и возможными финансовыми последствиями нарушения безопасности, при этом важно определить конкретные требования к безопасности в соответствии с услугами, подлежащими защите.

Требования по обеспечению безопасности сетей электросвязи включают:
\begin{itemize}
\item организационные требования безопасности;
\item технические требования безопасности;
\item функциональные требования безопасности;
\item требования доверия к безопасности.
\end{itemize}

ОТБ содержат общие организационные, административные положения и процедуры по осуществлению мероприятий политики безопасности оператором связи.

ТТБ определяют требования к электропитанию, заземлению, к конструкции средств связи, к линейно-кабельным сооружениям связи, к прокладке линий связи и др., влияющие на обеспечение безопасности и устойчивости функционирования сетей электросвязи.

ФТБ и ТДБ содержат требования, определенные ГОСТ Р ИСО/МЭК 15408-2 и ГОСТ Р ИСО/МЭК 15408-3 соответственно, которые для сетей и средств связи излагаются в профилях защиты и заданиях по безопасности и должны реализовываться на всех этапах жизненного цикла сетей электросвязи.
\subsection{Основные мероприятия по обеспечению безопасности СЭС}
\label{sec-1_6}


Обеспечение безопасности сети электросвязи является обязанностью ее владельца. Ответственность владельца сети электросвязи за обеспечение ее безопасности не прекращается при делегировании им своих полномочий по данным функциям отдельным лицам (поставщикам услуг, администраторам, третьим лицам и т.д.).

Мероприятия по обеспечению безопасности сети электросвязи, проводимые оператором связи, не должны ухудшать качественных характеристик сети и снижать оперативность обработки информации. Реализация обязательных требований к безопасности, установленных федеральными органами исполнительной власти в области связи, осуществляется силами и средствами владельца сети электросвязи с привлечением при необходимости специализированных организаций, имеющих лицензии на данный вид деятельности.

Дополнительные (повышенные) требования к безопасности (например шифрование трафика пользователя) могут осуществляться оператором связи на договорной основе с пользователем.

Вопросы непосредственного обеспечения безопасности при присоединении одной сети электросвязи к другой и условия выполнения обязательных требований к безопасности, установленные федеральными органами исполнительной власти в области связи, при взаимодействии этих сетей оговариваются в заключаемых операторами связи договорах о присоединении сетей электросвязи.

При присоединении к сетям электросвязи иностранных государств и взаимодействии с глобальными информационно-телекоммуникационными сетями, в том числе и Интернет, обеспечение безопасности должно основываться на соблюдении международных правовых актов, регламентирующих безопасный пропуск трансграничного трафика. При этом должна быть обеспечена защита инфокоммуникационной структуры сетей электросвязи от НСД со стороны взаимодействующих сетей и гарантированное качество обслуживания в условиях возможных ВН трансграничного характера.

Обеспечение безопасности сетей электросвязи достигается:
\begin{enumerate}
\item защитой сетей электросвязи от НСД к ним и передаваемой посредством их информации;
\item противодействием техническим разведкам;
\item противодействием сетевым атакам и вирусам;
\item защитой средств связи и сооружений связи от НСВ, включая физическую защиту сооружений и линий связи;
\item разграничением доступа пользователей и субъектов инфокоммуникационной структуры сетей электросвязи к информационным ресурсам в соответствии с принятой политикой безопасности оператора связи;
\item использованием механизмов обеспечения безопасности;
\item физической и инженерно-технической защитой объектов инфокоммуникационной структуры сетей электросвязи;
\item использованием организационных методов, включающих:

\begin{itemize}
\item разработку и реализацию политики безопасности оператором связи;
\item организацию контроля состояния безопасности сети электросвязи;
\item определение порядка действий в чрезвычайных ситуациях и в условиях чрезвычайного положения;
\item определения порядка реагирования на инциденты безопасности;
\item разработку программ повышения информированности персонала сети электросвязи в вопросах понимания им проблем безопасности;
\item определение системы подготовки и повышения квалификации специалистов в области безопасности.
\end{itemize}

\end{enumerate}

Пользователи услугами связи имеют право применять специальные механизмы обеспечения безопасности и СЗИ, разрешённые к применению на СЭС и сертифицированные в соответствии с действующим законодательством РФ. 

Взаимоотношения пользователей с операторами связи в сфере обеспечения безопасности СЭС должны строится на основании следующих положений:
\begin{itemize}
\item только авторизованные пользователи должны иметь доступ к сетям электросвязи и использованию предоставляемых им услуг;
\item авторизованные пользователи должны иметь доступ и оперировать только теми ресурсами, к которым они допущены;
\item все пользователи должны быть ответственными за их собственные, и только их собственные, действия в сети электросвязи.
\end{itemize}

Оператор связи должен принимать меры, обеспечивающие:
\begin{itemize}
\item доступ правоохранительных органов, в предусмотренных законодательством Российской Федерации случаях, к информации конкретных пользователей;
\item право на доступ пользователей услугами связи к информационным ресурсам в строгом соответствии с установленными правилами разграничения доступа;
\item исключение несанкционированного доступа пользователей услугами связи к ресурсам сети и услугам связи;
\item предоставление пользователям услугами связи дополнительных услуг по защите информации и процесса безопасной передачи сообщений на договорной основе;
\item информирование пользователей о состоянии безопасности доступа к услугам связи.
\end{itemize}
\subsection{Основные положения о структуре системы обеспечения безопасности сетей электросвязи}
\label{sec-1_7}


Система обеспечения безопасности (СОБ) сетей электросвязи ССОП является элементом системы информационной безопасности Российской Федерации и может быть отнесена к категории технологических систем связи.

Архитектура СОБ сетей электросвязи имеет многоуровневую иерархическую структуру, охватывающую магистральные транзитные, междугородние и зоновые (местные и внутризоновые) сети электросвязи, и состоит из взаимодействующих между собой служб обеспечения безопасности различных операторов связи, координируемых центральным органом СОБ, который может быть образован федеральным органом исполнительной власти в области связи.

Архитектура СОБ сети электросвязи может состоять из нескольких уровней безопасности, характеристика которых должна быть отражена в политике безопасности организации связи. В общем случае архитектура СОБ может содержать следующие уровни безопасности:

\begin{enumerate}
\item уровень управления безопасностью. На данном уровне осуществляется управление безопасностью сетей электросвязи, координируемое центральным органом СОБ;
\item организационно-административный уровень. Включает службы (отделы, подразделения, администраторов) безопасности, в зависимости от структуры организации связи. На данном уровне осуществляются:

\begin{itemize}
\item взаимодействие с системой управления сетями электросвязи;
\item управление, координация и контроль проводимых организационных и технических мероприятий на всех нижележащих уровнях;
\item учет практического применения нормативной правовой базы (законов, стандартов, положений, должностных инструкций, планов по безопасности);
\end{itemize}

\item уровень безопасности инфокоммуникационной структуры. Содержит механизмы обеспечения безопасности и другие средства, обеспечивающие защиту процесса обработки и передачи информации в сети. На данном уровне осуществляются:

\begin{itemize}
\item разграничение доступа к информационным ресурсам, сетевым объектам и системе управления сетью электросвязи,
\item защита от НСД, аутентификация и идентификация участников сетевого взаимодействия, включая удаленные объекты и администраторов (сетевых и безопасности),
\item контроль трафика (межсетевые экраны), средства обнаружения атак, средства регистрации и учета событий и ресурсов (аудит и мониторинг безопасности);
\end{itemize}

\item уровень безопасности услуг. На данном уровне осуществляется контроль качества обслуживания (предоставляемых услуг связи) в условиях возможных ВН и в чрезвычайных ситуациях, в том числе целостности циркулирующих в сети сообщений, содержащих данные пользователя и информацию управления;
\item уровень сетевой безопасности. Данный уровень поддерживает безопасность сетевых протоколов, которые обеспечивают:

\begin{itemize}
\item передачи трафика из конца в конец,
\item транспортирование файлов,
\item поддержку фундаментальных приложений, передачу голоса в сети и электронную почту;
\item конфиденциальность передаваемой по каналам связи информации управления;
\end{itemize}

\item уровень физической безопасности. На данном уровне обеспечиваются:

\begin{itemize}
\item физическая охрана помещений, в которых обрабатывается и хранится информация,
\item организация контроля доступа сотрудников и посетителей на территорию организации связи, в помещения со средствами связи, осуществляющими обработку информации, к технологическим системам управления, кабельным соединениям,
\item организация охранной сигнализации,
\item контроль вскрытия аппаратуры,
\item электро- и пожаробезопасность организации связи в целом.
\end{itemize}

\end{enumerate}

Оператор связи в целях обеспечения своей деловой деятельности и достижения бизнес-целей может определить дополнительные архитектурные компоненты СОБ.

Процедура создания СОБ сети электросвязи должна предусматривать формирование организационно-штатной структуры (отдел, подразделение, администратор безопасности) для непосредственного проведения мероприятий безопасности сети электросвязи.
\section{Домашняя работа}
\label{sec-2}
\subsection{Модель угроз безопасности для корпоративной сети связи ВУЗА}
\label{sec-2_1}
\subsubsection{Ресурсы инфокоммуникационной структуры СЭС, требующие защиты:}
\label{sec-2_1_1}


\begin{enumerate}
\item Абонентская база данных в памяти коммутатора.
\item Программное обеспечение АТС.
\item Аппаратная часть АТС.
\item Абонентская сеть связи.
\end{enumerate}
\subsubsection{Источники формирования дестабилизирующих воздействий и их потенициальные возможности:}
\label{sec-2_1_2}


\begin{enumerate}
\item \textbf{Производитель АТС}. Является специалистом высшей квалификации, знает все возможности АТС и, в частности, о системе и средствах ее защиты и скрытых возможностях. Не имеет физического доступа в КЗ, но может осуществить удалённый доступ по недекларированному каналу к АТС.
\item \textbf{Террорист}. Не является абонентов сети, не обладает знаниями о функционировании АС.
\item \textbf{Сотрудник университета}. Имеет общие представления о функционировании сети связи, имеет доступ к штатным средствам сети связи (может совершать звонки).
\item \textbf{Сотрудник университета, обслуживающий АТС}. Является специалистом высшей квалификации, знает все об АТС и, в частности, о системе и средствах ее защиты. Имеет доступ в контролируемую зону --- к аппаратной части АТС. Имеет доступ к утилитам администрирования и конфигурирования системы.
\end{enumerate}
\subsubsection{Описание возникновения угрозы:}
\label{sec-2_1_3}


\begin{enumerate}
\item Производитель заложил в АТС незадокументированную возможность удалённого доступа, которая позволяет дистанционно отлаживать неисправную систему в тех условиях, в которых она неисправно работает. Она также дает возможность дистанционно обновлять системы с обнаруженными дефектами. Это наиболее опасная уязвимость, т.к. доступ злоумышленника к программному обеспечению дает практически неограниченный доступ к АТС и сети.
\item Сотрудник университета, обслуживающий АТС, узнаёт о своём сокращении и решает отомстить руководству Университета. Он использует штатную утилиту проверки/модификации станционной базы данных: такая утилита позволяет исследовать и модифицировать базу данных системы для устранения неисправностей из-за неправильной конфигурации, ошибки конструкции и т.п. Он меняет маршрутизацию в сети. В результате работа сети нарушена.
\item Террорист с целью самоутверждения, придания своей деятельности особой значимости проникает через проходную Университета и закладывает бомбу рядом с комнатой, где расположена АТС. Реализована атака типа отказ в обслуживании.
\item Сотрудник университета в корыстных целях подключает устройство записи к абонентской линии ректора Университета и ведёт прослушивание конфиденциальных переговоров с целью перепродажи данных сведений заинтересованным лицам.
\end{enumerate}
\subsubsection{Стадии жизни цикла СЭС:}
\label{sec-2_1_4}


\begin{enumerate}
\item Предпроектный анализ.
\item Проектирование системы.
\item Разработку системы.
\item Интеграцию и сборку системы, проведение ее испытаний.
\item Эксплуатацию системы и ее сопровождение.
\item Развитие системы.
\end{enumerate}
\section{ГОСТ 15408}
\label{sec-3}
\subsection{Основные понятия, общие критерии (ОК)}
\label{sec-3_1}


ОК содержат 2 основных требования вида безопасности:
\begin{itemize}
\item функциональные, соответствующие активному аспекту защиты, предъявляемые к функциям безопасности и реализующим им механизмам;
\item требования доверия, соответствующие пассивному аспекту, предъявляемые к технологии и процессу разработки и эксплуатации.
\end{itemize}

Требования безопасности формулируются и их выполнение проверяется для определённого объекта оценки (ОО), т. е. аппаратно-программного продукта ИТ или системы ИТ.

Безопасность в ОК рассматривается на жизненном цикле ОО.

Кроме того, объект оценки рассматривается в контексте среды безопасности, характеризующейся определёнными условиями и угрозами. Требования в общих критериях формулируются в документах 2 видов:
\begin{itemize}
\item профиля защиты (ПЗ). Типовой набор требования, которым должны удовлетворять продукты и (или) системы определённого класса;
\item задания по безопасности (ЗБ). Содержит совокупность требований к конкретной разработке продукта или системы.
\end{itemize}

\emph{Системой ИТ} называется специфичная реализация ИТ с конкретным назначением и условиями эксплуатации.

\emph{Продукт ИТ} представляет собой совокупность средств ИТ, предоставляющих определённые функциональные возможности и предназначенных для непосредственного использования либо включения в различные системы. Продукт или система могут быть уже существующими или проектируемыми.

В среду безопасности объекта оценки включаются:
\begin{enumerate}
\item Законодательная среда (нормативные акты, затрагивающие объекты оценки).
\item Административная среда (положения политик и программ безопасности, учитывающие особенности объекта оценки).
\item Процедурная среда (физическая среда объекта оценки и меры и его физической защиты, персонал и его свойства, принятые эксплуатационные и иные процедуры).
\item Программно-техническая среда (предназначение объекта оценки и предполагаемая область его применения, активы (ресурсы, которые требуют защиты объектами оценки)).
\end{enumerate}

Из анализа среды безопасности должны быть описаны следующие объекты:
\begin{enumerate}
\item Предположение безопасности, которое выделяет объект оценки из общего контекста, задаёт границы рассмотрения. Истинность этих предположений принимается без доказательств, а из множества возможных отбираются только те, что заведомо необходимы для обеспечения безопасности объект оценки.
\item Угрозы безопасности объекту оценки, наличие которых в рассматриваемой среде установлено или предполагается. Они характеризуются следующими параметрами:

\begin{itemize}
\item источник;
\item метод воздействия;
\item опасные с точки зрения закономерности использования уязвимости;
\item активы, потенциально подверженные повреждению. При анализе рисков угроз принимается во внимание вероятность активации угрозы и её успешного осуществления, а также размер возможного ущерба. По результатам анализа из множества допустимых угроз отбираются только те, ущерб от которых нуждается в уменьшении.
\end{itemize}

\item Положения политики безопасности, предназначенные для применения к объекту оценки. Для системы ИТ такие положения могут быть описаны точно, для продукта ИТ в общих чертах.
\end{enumerate}

На основании положений об учёте угроз и положений политики безопасности формулируются цели безопасности для объекта оценки, направленные на обеспечение противостояния угрозам и выполнение политики безопасности. В зависимости от непосредственного отношения к объекту оценки или среде, они делятся на цели безопасности объекта оценки и цели безопасности среды.

Общие критерии, а именно 2 и 3 части являются каталогами требований безопасности. В основу методологии общих критериев положена модель безопасности, представленная на рисунке.

Для структуризации простраства требований в ОК введения иерархия Класс - Семейство - Компонент - Элемент.

Классы определяют наиболее общую группировка требований. Семейства в пределах класса различаются по строгости и другим характерстикам. Компонент определяется минимальным набором требований, фигурирующим как единое целое. Элемент --- это неделимое требование к безопасности.

Между критериями введены зависимости, когда компонент сам по себе недостаточен для достижения целей безопасности. После формулирования функциональных требования, требований доверия к объекту оценки и его среде в ПЗ и ЗБ можно приступасть к оценке безопасности продукта или системы.

ПЗ от ЗБ отличается двумя разделами. В ЗБ добавляются краткая спецификация объекта оценки и утверждение о соответствии профилю защиты.

Профиль защиты включает в себя следующие разделы:
\begin{enumerate}
\item Введение, состоящее из подразделов идентификации ПЗ и аннотации ПЗ.
\item Описание объекта оценки.
\item Среда безопасности объекта оценки, состоящий из подразделов предположения безопасности, угроз, политик безопасности организации.
\item Цели безопасности, состоящие из подразделов целей безопасности для объекта оценки и целей безопасности для среды.
\item Требования безопасности ИТ, состоящие из требований безопасности для объекта оценки, включая функциональные требования, требования доверия безопасности к объекта оценки и требования безопасности для среды ИТ.
\item Замечания по применению и обоснование, состоящее из подразделов логического обоснования требований безопасности и логического обоснования целей безопасности. В ЗБ дополнительно имеются следующие разделы:

\begin{itemize}
\item краткая спецификация объекта оценки, состоящая из функций безопасности объекта оценки и спецификации мер доверия;
\item утверждение соответствии профилю защиты, в котором приводится ссылка на ПЗ, конкретизация ПЗ и дополнения ПЗ.
\end{itemize}

\end{enumerate}

Раздел введения дополняется разделом соответствия ОК. В раздел обоснования добавляются подраздел логического обоснования, краткая спецификация объекта оценки и логического обоснования утверждения о соответствии ПЗ. Краткая спецификация определяет отражение требования на функции безопасности.

Общие критерии не предписывают общей методологии или дисциплины разработки модели ИТ, но предусматривают наличие нескольких уровней представления проекта с его декомпозицией и детализацией.

За требованиями безопасности следует функциональная спецификация, затем проект верхнего уровня, необходимое число промежуточных уровней, проект нижнего уровня, исходный код или схема аппаратура и реализация в виде исполняемых файлов, программных продуктов и т. п.

Между уровнями представления должно демонстрироваться соответствие, т. е. все сущности более высоких уровней обязаны фигурировать и ниже. А внизу не должно быть место лишним сущностям, не обусловленным потребностями более высоких уровней.

При проведении оценки главными являются следующие вопросы:
\begin{enumerate}
\item Отвечают ли функции безопасности объекта оценки функциональным требованиям.
\item Конкретна ли реализация функции безопасности.
\end{enumerate}

Если оба ответа положительны, то говорят о достижении целей безопасности.
\subsection{Классификация функциональных требований безопасности}
\label{sec-3_2}


Часть 2 общих критериев описывает 11 классов, 66 семейств, 35 компонентов ФТБ и содержат требования о том, какие цели безопасности могут быть достигнуты при современном уровне ИТ и каким образом.

Функциональные компоненты могут быть не до конца конкретизированы в ОК, поэтому фактические параметры подставляются в ПЗ и ЗБ. Такая операция называется назначением.

В качестве параметров могут выступать, например такие сложные сущности, как политика безопасности.

Некоторые компоненты в ОК задаются с ``запросом''. В них включается список возможностей, из которых потом осуществляется выбор той, что необходима в конкретной ситуации. Например обнаружение и/или предотвращение определённых положений политики безопасности.

Любой функциональный компонент допускает операции по многократному использованию, например для охвата различных аспектов объекта оценки, называемые в ОК итерациями, а также уточнение и добавление дополнительных деталей.

Между компонентами ФТБ могут существовать зависимости. Они возникают, когда компонент не является самодостаточным и для своей реализации нуждается в привлечении других компонентов.

Классы ФТБ можно условно разделить в зависимости от того, описывают ли они элементарные сервисы безопасности или производные, реализуемые на основе элементарных; направлены ли они на достижение высокоуровневых целей безопасности или играю  инфраструктурную роль.

К первой группе можно отнести следующие классы:
\begin{enumerate}
\item FAU. Аудит безопасности.
\item FIA. Идентификация и аутентификация.
\item FRU. Использование ресурсов.
\end{enumerate}

Класс FAU состоит из 6 семейств, содержащих требования к отбору, регистрации, хранению и анализу данных о действиях и событиях, затрагивающих безопасность объекта оценки.

Класс FIA состоит из 6 семейств, содержащих требования к идентификация пользователей, аутентификации пользователей, определению атрибутов пользователя, связыванию пользователя с субъектом, к отказыванию от аутентификации и спецификации секретов.

Класс FRU включает 3 семейства, призванные разными способами поддерживать высокую доступность:
\begin{itemize}
\item отказоустойчивость,
\item приоритет обслуживания,
\item распределение ресурсов.
\end{itemize}

Ко 2 группе можно отнести следующие классы:
\begin{enumerate}
\item FCO. Связь.
\item FPR. Приватность.
\end{enumerate}

Класс FCO состоит из 2 семейств неотказуемость отправки или получения данных, которая достигается путём избирательной или принудительной генерации, допускающих верификацию свидетельств, позволяющих ассоциировать атрибуты отправителя (получателя) с элементами передаваемых данных.

Класс FPR содержит 4 семейства, обеспечивающих защиту пользователя от раскрытия и несанкционированного использования его идентификационных данных:
\begin{itemize}
\item анонимности,
\item псевдонимность,
\item невозможность ассоциаций,
\item скрытность.
\end{itemize}

Достичь высокоуровневых целей безопасности помогают 2 класса:
\begin{enumerate}
\item FDP. Защита данных пользователя.
\item FPT. Защита функций безопасности объекта оценки.
\end{enumerate}

Класс FDP включает 13 семейств, которые можно разбить на 4 группы:
\begin{enumerate}
\item Политики защиты данных пользователя.
\item Виды защиты данных пользователя.
\item Импорт и экспорт данных пользователя.
\item Защита данных пользователя при передаче между доверенными продуктами и системами ИТ.
\end{enumerate}

Класс FPT включает 16 семейств, которые можно условно разделить на 4 группы:
\begin{enumerate}
\item Архитектурная безопасность.
\item Защита реализаций функций безопасности.
\item Защита данных функций безопасности.
\item Инфраструктурные требования.
\end{enumerate}

Наибольшее число компонентов сосредоточены в классах инфраструктурной группы.
\begin{enumerate}
\item FCS. Криптографическая поддержка.
\item FMT.
\item FTA. Доступ к объекту оценки.
\item FTP. Доверенный маршрут канала.
\end{enumerate}

Класс FCS состоит из 2 семейств, где в самом общем виде рассматривается генерация, распределение, доступ и уничтожение ключей, а также криптографические операции. Смысл требований состоит в том, что необходимо действовать в соответствии с некими алгоритмами длинами ключей и стандартами. Какие либо содержательные методики отсутствуют.

Класс FMT, включает 16(?) семейств регулирует управление функциями безопасности и из данными атрибутами и ролями безопасности.

Класс FTA содержит 6 семейств, в которые вошли требования управления сеансами работы пользователей (помимо идентификации и аутентификации).

Класс FTP, состоящий из 2 семейств доверенный маршрут и доверенный канал, обеспечивает требования по созданию маршрутов/каналов передачи информации безопасным способом.

Пример описания функциональных требования. Рассмотрим описание класса, семейства, компонента элемента требований на примере класса FCO связь. Класс FCO содержит 2 семейства, связанные с уверенностью в идентичности сторон, участвующих в обмене данными. Идентичностью отправителя переданной информации (доказательства отправления) и идентичностью получателя переданной информации (доказательства получения). Эти семейства обеспечивают, что отправитель не сможет отрицать факт оправления сообщения, а получатель не сможет отрицать факт его получения. Декомпозиция класса на составляющие его компоненты показана на рисунке.

Семейство FCO$_{\mathrm{NRO}}$ обеспечивает невозможность отрицания отправителем информации факта её отправления. Семейство содержит требования, чтобы функции безопасности объекта оценки обеспечили метод предоставления субъекту получателю свидетельства оправления информации. Это свидетельство может быть верифицировано этим субъектом или другими субъектами.

Компоненты внутри семейства проранжированы иерархически последовательно. FCO$_{\mathrm{NRO}}$.1 (избирательное доказательство отправления) содержит требования чтобы функции безопасности объекта оценки предоставили субъектам возможность запросить свидетельства отправления информации. FCO$_{\mathrm{NRO}}$.2 (принудительное доказательство отправления) содержит требования, чтобы функции безопасности объекта оценки всегда генерировали свидетельства отправления передаваемой информации.

Управление: для функций управления для класса FMT может рассматриваться следующие действия.
\begin{itemize}
\item управление изменениями типов и полей информации, атрибутов отправителя информации и получателей свидетельств,
\item аудит:  FCO$_{\mathrm{NRO}}$.1 --- если в ПЗ или ЗБ включено семейство FAU$_{\mathrm{GEN}}$ генерация данных аудита безопасности, то следует предусмотреть возможность (в зависимости от выбранного уровня) аудита следующих действий/событий/параметров

\begin{itemize}
\item минимальный: идентификатор пользователя, который запросил генерацию свидетельства отправления, обращение к функциям неотказуемости
\item базовый: идентификатор информации, получателя и копии предоставляемого свидедельства.
\item детализированный: идентификатор пользователя, который запросил  верификацию свидетельства.
\end{itemize}

\item аудит: FCO$_{\mathrm{NRO}}$.2 --- если в ПЗ или ЗБ включено семейство FAU$_{\mathrm{GEN}}$ генерация данных аудита безопасности, то следует предусмотреть возможность (в зависимости от выбранного уровня) аудита следующих действий/событий/параметров:

\begin{itemize}
\item минимальный: обращение к функции неотказуемости.
\item базовый: идентификация информации, получателя и копии предоставляемого свидетельства.
\item детализированый: идентификатор пользователя, который запросил верификацию свидетельства.
\end{itemize}

\end{itemize}

Описание компонента FCO$_{\mathrm{NRO}}$.1 избирательное доказательство отправления выглядит следующим образом:
\begin{itemize}
\item иерархический для FCO$_{\mathrm{NRO}}$.1: нет подчинённых компонентов. Элементы компонента FCO$_{\mathrm{NRO}}$.1 описаны ниже.

\begin{itemize}
\item FCO$_{\mathrm{NRO}}$.1.1 FBO функции безопасности объекта оценки должны быть способны генерировать свидетельство отправления передаваемой. [ \emph{Список типов информации} ]. Передаваемой при забросе [выбор: отправитель-получатель,] [ \emph{назначение}: список третьих лиц]
\item FCO$_{\mathrm{NRO}}$.1.2 FBO должны быть способны связать [назначение: список атрибутов] отправителя информации и [назначение: список информационных полей] информации, к которой прилагается свидетельство.
\item FCO$_{\mathrm{NRO}}$.1.3 FBO должны предоставить возможность верифицировать свидетельство отправления информации [выбор: отправитель, получатель [назначение: список третьих лиц]] при установленных [назначение: ограничение на свидетельство отправления].
\end{itemize}

\end{itemize}

Зависимости данного компонента --- FIA$_{\mathrm{UID}}$.1 выбор момента идентификации.
\subsection{Основные понятия, классификация требования доверия безопасности}
\label{sec-3_3}


Доверие в интерпретации ОК --- это основа для уверенности в том, что продукт или система ИТ отвечает целям безопасности.

Доверие обеспечивается через активные исследование/оценку продукта или системы. Требования доверия безопасности (ТДБ) охватывают весь жизненный цикл объекта оценки и предполагают выполнение следующих действий:
\begin{enumerate}
\item Оцениваются ЗБ и ПЗ как источники требований безопасности.
\item Анализируются различные представления проекта объекта оценки и соответствия между ними, а также соответствия каждого из них требованиям безопасности.
\item Проверяются процессы и процедуры безопасности, их применение, анализируется документация, верифицируются представленные доказательства.
\item Анализируются тесты и их результаты, а также уязвимости объекта оценки.
\item Проводятся независимые тестирования, в т. ч. тестирование проникновения.
\end{enumerate}

Каждое требование (элемент доверия) принадлежит одному из трёх типов:
\begin{enumerate}
\item Элементы действий разработчика (помечаются буквой D после номера элемента). Эти действия должны подтверждаться доказательственным материалом (свидетельством).
\item Элементы представления и содержания свидетельств (помечаются буквой S).
\item Элементы действия оценщика (помечаются буквой E).
\end{enumerate}

Оценщики обязаны проверить представленные разработчиками свидетельства, а также выполнить необходимые дополнительные действия, например провести независимое тестирование.

Требования доверия разделены на 10 классов, 44 семейства, 93 компонента.

Классы можно сгруппировать в зависимости от охватываемых этапов жизненного цикла объекта оценки.

К первой группе, логически предшествующей разработке и оценке объекта оценки принадлежать классы:
\begin{enumerate}
\item APE оценка профиля защиты.
\item ASE оценка задания по безопасности.
\end{enumerate}

Цель требований классов APE и ASE проверить полноту, непротиворечивость и реализуемость ПЗ или ЗБ.

Во вторую группу входят классы:
\begin{enumerate}
\item ADV разработка.
\item ALC поддержка жизненного цикла.
\item ACM управление конфигурацией.
\end{enumerate}

Класс ADV состоит из 7 семейств и содержит требования для постепенного повышения уровня детализации проекта вплоть до предоставления реализаций с демонстрацией соответствия между уровнями. В этом классе предусмотрено 3 стиля изложения спецификации: неформальный, полуформальный и формальный --- и 3 способа демонстрации соответствия.

Технологические требования процедурного характера составляют содержание класса ALC, состоящего из 4 семейств. Прежде всего определяется модель жизненного цикла (семейства ALC$_{\mathrm{LCD}}$), затем следует обосновать выбор инструментальных средств и методов (семейства ALC$_{\mathrm{TAB}}$). Безопасность разработки организуется в соответствии с требованиями семейства ALC$_{\mathrm{DVC}}$.
Важнейшим элементом этапа сопровождения является устранение недостатков (семейство ALC$_{\mathrm{FLR}}$). 

Управление конфигурацией ACM --- необходимый инструмент коллектива разработчиков. В этот класс входит 3 семейства. Самый содержательный из них --- ACM$_{\mathrm{CAB}}$, специфицирующие возможности управления конфигурацией. Семейство ACM$_{\mathrm{SCP}}$ специфицирует область действий управления конфигурацией. Для уменьшения числа возможных ошибок управление конфигурацией следует максимально автоматизировать. В этом смысл требований семейства ACM$_{\mathrm{AOT}}$.

К этапу получения, представления и анализа результатов разработки можно отнести классы AGD --- руководство пользователя, администратора, ATE --- тестирование, AVA --- оценка уязвимостей.

Класс AGD состоит из 2 семейств, где сформулированы требования к руководству администратора AGD$_{\mathrm{ADM}}$, руководство пользователя AGD$_{\mathrm{USR}}$.

Класс ATE состоит из 3 семейств, содержащих требования к полноте, глубине, способам и результатам тестирования функций безопасности на предмет из соответствия спецификациям.

Один из ключевых моментов оценки безопасности продуктов ИТ --- оценка уязвимостей, отправным пунктом которой является анализ уязвимостей (семейства AVA$_{\mathrm{VLA}}$), выполняемый разработчиком и оценщиком. Анализ стойкости функций безопасности объекта оценки (семейство AVA$_{\mathrm{SOF}}$) проводится на уровне реализующих механизмов.

Требования семейство AVA$_{\mathrm{MSV}}$ (неправильное применение) направлены на то, чтобы исключить возможность такого конфигурирования и/или применения объекта оценки, которая администратор или пользователь считает безопасным в то время, как оно таковым не является.

Анализ скрытых каналов, регламентируемый семейством AVA$_{\mathrm{CCA}}$ требует, чтобы разработчик проводил исчерпывающий поиск скрытых каналов для каждой политики управления политики управления информационными потоками и предоставлял документацию анализа, а оценщик должен выборочно подтвердить правильность анализа скрытых каналов посредством тестирования.

Класс ADO поставка и эксплуатация содержит требования к процедурам поставки, установки, генерации и запуска объекта оценки.

Класс AMA поддержка доверия включает требования, применяемые после сертификации объекта оценки на соответствие общим критериям. Они помогают по возможности экономно, без полной повторной оценки сохранять уверенность в том, что объект оценки продолжает отвечать своему заданию по безопасности после изменений в нём или в его среде. Речь идёт о выявлении новых угроз и уязвимостей, изменений в требованиях пользователей об исправлении ошибок.

Компоненты требования доверия линейно упорядочены в пределах семейства, т. е. компонент с большим номером всегда усиливает предыдущий.

Одна из целей общих критериев состоит в минимизации усилий оценщиков и разработчиков, направленных на обеспечение заданного уровня доверия. Этому способствует введение семи оценочных уровней доверия (ОУД), содержащих полезные для практического применения комбинации компонентов, упорядоченные по степени усиления.

Повысить уровень доверия помогают дополнительные действия:
\begin{enumerate}
\item Расширение границ объекта оценки.
\item Увеличение уровня детализации рассматриваемых аспектов объекта оценки.
\item Повышение строгости рассмотрения и применение более формальных методов верификации.
\end{enumerate}
\subsection{Оценочный уровень доверия безопасности}
\label{sec-3_4}


В общих критериях определено 7 упорядоченных по возрастанию ОУД, содержащих рассчитанные на многократное применение комбинации требований доверия (не более 1 компонента из соответствующего семейства). Наличие такой шкалы даёт возможность сбалансированного получения уровней доверия со сложностью, сроками, стоимостью и самой возможностью их достижения.

Предполагается, что в ПЗ и ЗБ будут фигурировать или сами ОУД, или их усиления, полученные путём расширения требований (за счёт добавления к ОУД новых компонентов), либо увеличения строгости, и/или глубины оценки (посредством замены компонентов более сильным вариантом из того же семейства).

В ОУД не включены требования классов OPE, OSE, OMA, поскольку они находятся за пределами основного цикла разработки продуктов и систем ИТ.

ОУД.1, предусматривающий функциональное тестирование применим, когда требуется некоторая уверенность, что объект оценки работает безукоризненно, а угрозы безопасности не считаются серьёзными. Его можно достичь без помощи разработчика и с минимальными затратами по средством анализа спецификации интерфейсов, эксплуатационной документации в сочетании с независимым тестированием.

ОУД.2, предусматривающий структурное тестированием и доступ к части проектной документации и результатам тестирования разработчиков применим, когда разработчикам или пользователям требуется независимо получаемый умеренный уровень доверия при отсутствии доступа к полной документации по разработке. 

В дополнение к ОУД.1 предписывается анализ проекта верхнего уровня. Анализ должен быть поддержан независимым тестированием функции безопасности, актом разработчика об испытаниях, основанных на функциональной спецификации, выборочном независимом подтверждении результатом тестирования разработчика, анализом стойкости функций безопасности и свидетельстве поиска явных уязвимостей.

Требуется наличие списка конфигураций объекта оценки с уникальной идентификацией элементов конфигурации и свидетельства безопасных процедур поставки.

ОУД.3 предусматривающий систематическое тестирование и проверку позволяет достичь максимально возможного доверия при использовании обычных методов разработки. Он применим в тех случаях, когда разработчики или пользователя требуется умеренный уровень доверия на основе всестороннего исследования объекта оценки и процесса его разработки. По сравнению с ОУД.2 сюда добавлено требования, которые предписывают разработчику создавать акт об испытаниях с учётом особенностей не только функциональной спецификации, но и проекта верхнего уровня, кроме того, требуется контроль среды разработки управления конфигурацией объекта оценки.

ОУД.4 предусматривающий систематическое проектирование, тестирование и просмотр позволяет достичь доверия максимально возможного при следовании общепринятой практики коммерческой разработки. Это самый высокий уровень, по которому вероятно экономически целесообразно ориентироваться для существующих типов продуктов.

ОУД.4 характеризуется анализом функциональной спецификации, полной спецификацией интерфейсов, эксплуатационной документацией, проектами верхнего и нижнего уровней, а также подмножеством реализаций применения неформальной модели политики безопасности объекта оценки. Среды других дополнительных требований выделяют независимый анализ уязвимостей, демонстрирующий устойчивость к попыткам проникновения нарушителей с низким потенциалом нападения и автоматизацию управления конфигурацией. Отличительной особенностью ОУД.5 --- это полуформальное проектирование и тестирование. С его помощью достигается доверие, максимально возможное при следовании строгой практики коммерческой разработки, поддержанной умеренным применением специализированных методов обеспечения безопасности.

ОУД.5 востребован, когда нужен высокий уровень доверия и строгий подход к разработке, не влекущий излишних затрат. Для достижения ОУД.5 требуется формальная модель политики безопасности объекта оценки, полуформальное представление функциональной спецификации и проект верхнего уровня, полуформальная демонстрация соответствия между ними, а также модульная структура объекта оценки. Акт об испытаниях должен быть основан ещё и на проекте нижнего уровня. Необходима устойчивость к попыткам проникновения нарушителей с умеренным потенциалом нападения. Предусматривается проверка правильности анализа разработчиком скрытых каналов и всестороннего управления конфигурацией.

ОУД.6 характеризующийся полуформальной верификацией проекта, позволяет получить высокое доверие путём применения специальных методов проектирования в строго контролируемой среде разработки при производстве высококачественных продуктов ИТ и при защите ценных активов от значительных рисков.

Особенности ОУД.6:
\begin{itemize}
\item структурированное представление реализации;
\item полуформальное представление проекта нижнего уровня;
\item иерархическая структура проекта объекта оценки;
\item устойчивость к попыткам проникновения нарушителей с высоким потенциалом нападения;
\item проверка правильности систематического анализа разработчиком скрытых каналов;
\item использование структурированного процесса разработки;
\item полная автоматизация управления конфигурацией объекта оценки.
\end{itemize}

ОУД.7 предусматривающий формальную верификацию проекта применим к разработке продуктов ИТ для использования в ситуациях чрезвычайно высокого риска или там, где высокая ценность активов оправдывает повышенные затраты. На 7 уровне дополнительно требуется:
\begin{itemize}
\item формальное представление функциональной спецификации проекта верхнего уровня и формальная демонстрация соответствия между ними;
\item модульная, иерархическая и простая структура проект объекта оценки, добавление представления реализации как основы акта об испытаниях проекта;
\item полное независимое подтверждение результатов тестирования разработчиком.
\end{itemize}
\subsection{Основные понятия и идеи общей методологии и оценки (ОМО) безопасности ИТ. Входная и выходная задачи, задачи оценки}
\label{sec-3_5}


С целью унификации процедуры сертификации по ОК в августе 1999 года была опубликована общая методология оценки безопасности информационных технологий, описывающая минимальный набор действий при проведении оценки. Проекта ОК с самого начала носил не только технический, но и экономико-политический характер. Его цель состояла в частности в том, чтобы упростить, удешевить и ускорить выход сертифицированных изделий ИТ на мировой рынок. Для этого в мае 2000 года уполномоченная правительственной организацией 6 стран основателей проекта ОК, а также Австралией, Новой Зеландией, Финляндией, Швецией, Грецией\ldots{} подписали соглашение о признании сертификатов по ОК в области безопасности ОТ. Участие в соглашении предусматривает соблюдение 2 независимых условий:
\begin{itemize}
\item признание сертификатов, выданных соответствующими органами других стран-участниц;
\item возможность осуществления подобной сертификации.
\end{itemize}

Очевидно, что от взаимного признания сертификатов выигрывают не только производители ИТ, но и потребители. Что же касается их выдачи, то соглашение предусматривает жёсткий контроль при получении и подтверждении этого права (например предусмотрено проведение т. н. теневых экспертов). Т. о. для полноценного участия в соглашении помимо желания государство должно располагать органами сертификации с достаточными ресурсами и штатом специалистов, квалификация которых получила официальное международное признание.

Основная цель ОМО --- добиться объективности, повторяемости и воспроизводимости. В процессе оценки выделяются задачи:
\begin{itemize}
\item входная задача;
\item задача оценки;
\item выходная задача.
\end{itemize}

\emph{Входная задача} имеет дело с представленными для оценки свидетельствами. Её назначение --- убедиться, что версии свидетельств корректны и должным образом защищены.

Обычно для оценки представляются стабильные, официально выпущенные версии свидетельств, однако, в ситуациях, когда оценка ведётся параллельно разработке или доработке объекта оценки, возможно предъявление рабочих версий. Оценщику вместе со спонсором этого процесса необходимо составить каталог и в дальнейшем производить конфигурационный контроль версий. Он обязан обеспечить защиту свидетельств от изменения и утери, а по окончании процесса оценки возвратить их, поместить в архив или уничтожить.

На всех этапах оценки должна обеспечиваться конфиденциальность.

\emph{Задача оценки} в общем случае разбивается на следующие подзадачи:
\begin{itemize}
\item оценка ЗБ,
\item оценка управления конфигурацией ОО,
\item оценка документации по передаче ОО потребителю и эксплуатационная документация,
\item оценка документации разработчиков,
\item оценка руководств,
\item оценка поддержки жизненного цикла объекта оценки,
\item оценка тестов,
\item оценка анализа уязвимостей.
\end{itemize}

Часто проводятся выборочные проверки, когда вместо всего множества свидетельств анализируется представительное подмножество, что позволяет сэкономить ресурсы при сохранении необходимого уровня доверия безопасности.

Размер выборки должен быть обоснован математически и экономически, но при реализации объекта оценки он должен составлять не менее 20\%. Ошибки, обнаруженные при выборочной проверке делятся на систематические и случайные.

После исправления систематической ошибки необходимо произвести новую выборку. После случайной этого не требуется.

Допускается выборочная проверка доказательств тестов, результатов анализа скрытых каналов, выполнение требований к содержанию и представлению свидетельств, выборочное тестирование. В остальных ситуациях такой способ можно применять только в исключительных случаях, когда полная проверка требует слишком много ресурсов по сравнению с другими действиями в процессе оценки или когда она не существенно увеличивает доверие безопасности. При этом необходимо обосновать допустимость и целесообразность такого подхода.

В ОМО специально подчёркивается, что сами по себе большие размеры и высокая сложность объекта оценки не оправдывает замены полных проверок выборочными, поскольку для оценки безопасности подобных объектов заведомо требуется много сил и средств.

Необходимый элемент проверки --- проверка внутренней согласованности каждого из представленных свидетельств, а также внешние взаимные согласованности различных свидетельств.

Внутренняя согласованность проверяется в первую очередь для сущностей, имеющих несколько представлений для спецификаций проекта всех уровней, для руководств.

Проверка внешней согласованности производится для описания функций, параметров безопасности, процедур и событий, связанных с безопасностью, поскольку эти описания могут содержаться в разных документах.

Внутренняя несогласованность высокоуровневых сущностей может иметь глобальные последствия для процесса оценки, например выявление противоречий в целях безопасности.

Это может заставить заново проанализировать требования к функциям безопасности.

Разные подзадачи в процессе оценки могут выполняться в произвольном порядке или параллельно, однако, существуют зависимости, накладывающие ограничения на очерёдность выполнения, например очевидно, что анализ ЗБ должен выполняться до каких бы то ни было проверок\ldots{}

ЗБ среди других характеристик ОО определяет его границы и спектр рассматриваемых угроз, следовательно, процесс и результат оценки одного и того же продукта в сочетании с разными ЗБ могут быть разными. Например, если в ОО содержатся средства межсетевого экранирования и поддержки виртуальных частных сетей, но в ЗБ предусмотрено исключительно защита внутренней сети от внешних устройств, то средство ВЧС функций важны в этом случае лишь в контексте возможности обхода средств экранирования.

Даже если ВЧС функции не обеспечивают конфиденциальность сетевых потоков данных, продукт с таким ЗБ получить положительную оценку.

Рассмотрим \emph{выходную задачу} --- её цель сформулировать замечания и получить технический отчёт оценки. Текст с замечаниями необязателен. Он нужен, если в процессе оценки выявились какие-либо неясности.

Технический отчёт оценки --- главный документ, от качества которого во многом зависит повторяемость и воспроизводимость оценки.

ОМО предписывает следующую структуру подобных отчётов:
\begin{itemize}
\item введение;
\item архитектурное (высокоуровневое описание объекта оценки с рассмотрением основных компонент);
\item описание процесса оценки, применённых методов, методологический инструмент, средства и стандарты;
\item представления результатов оценки, выводы и рекомендации;
\item список представленных свидетельств;
\item список сокращений, словарь терминов;
\item словарь замечаний.
\end{itemize}
\section{Система менеджмента ИБ в организациях электросвязи}
\label{sec-4}
\subsection{Процессный подход и модель СМИБ телекоммуникации}
\label{sec-4_1}


Для организации информация, вспомогательные устройства, сети и линии электросвязи являются важными активами бизнеса. Для должного управления этими активами бизнеса и для правильного и успешного продолжения бизнеса организации электросвязи чрезвычайно важно управление ИБ.

СМИБ предназначается для обеспечения достаточных и соразмерных средств управления безопасностью, которые адекватно защищают информационные активы и предают уверенность клиентам и деловым партнёрам организации электросвязи и также другим заинтересованным сторонам электросвязи.

Это может служить средством поддержания и улучшения конкурентоспособности, увеличения денежных потоков и доходности, соблюдения правовых норм и улучшение коммерческой репутации.

СМИБ --- это часть общей системы менеджмента, основанная на подходе бизнес-риска для установления, реализации, эксплуатации, мониторинга, анализа, обслуживания и усовершенствования ИБ.

Для эффективного функционирования организация электросвязи должна определять множество действий и управлять ими. Любое действие, использующее ресурсы и управляемое с целью создать возможность преобразования входных данных в выходные может рассматриваться как процесс.

Часто выходные данные одного процесса непосредственно образуют входные данные следующего процесса. Применение систем и процессов внутри организации совместно с идентификацией и взаимодействием этих процессов, а также управления ими, может быть названа процессным подходом, или подходом, основанном на процессах.

Процессный подход способствует акцентированию внимания его пользователей на важность:
\begin{itemize}
\item понимания требований бизнеса к ИБ и необходимости установления политики и целей ИБ;
\item реализации, эксплуатации средств контроля с точки зрения управления всеми рисками бизнеса организации;
\item контроля и анализ рабочих характеристик и эффективности СМИБ;
\item постоянного совершенствования, основанного на объективных измерениях.
\end{itemize}

Модель, известная как, планирование работы проекта(?) ПДЦА может быть применима ко всем проектам СМИБ.

Показано, как используется СМИБ для введения требований к ИБ и ожиданий организаций электросвязи и заинтересованных сторон и как путём необходимых действий и процессов создать выходные продукты ИБ, которые соответствуют этим требованиям и ожиданиям. Планирование и установление СМИБ создаёт политику безопасности, цели, задачи, процессы и процедуры, соответствующие управляемым рискам и улучшенной ИБ для представления результатом в соответствии с общей политикой и целями организации.

Осуществление, реализация, эксплуатация СМИБ реализует и применяет политику безопасности, средства управления, процесс и процедуры. 

Проверка, мониторинг и анализ СМИБ --- оценки и где это применимо, измерение рабочих характеристик процесса, относящегося к политике, целями безопасности и практическому опыту и представление отчёта о результатах систем управления для анализа.

Действий, поддержка и усовершенствование СМИБ --- принятие корректирующих и превентивных действий, основанных на результатах анализа руководством для достижения непрерывных усовершенствований СМИБ.
\subsection{Процессы СМИБ}
\label{sec-4_2}


Организация должна разрабатывать, реализовывать, поддерживать и непрерывно совершенствовать документированную СМИБ с позиции всей деловой деятельности и риска организации.

\emph{Создание СМИБ}. Организация должна:
\begin{itemize}
\item определить область применения СМИБ;
\item определить политику СМИБ;
\item определить системный подход к определению риска;
\item идентифицировать риски;
\item количественно определить риски;
\item идентифицировать и оценить варианты обработки рисков;
\item выбрать цели управления и средства управления для обработки рисков;
\item подготовить заявление о применимости;
\item получить согласие руководства на предлагаемые остаточные риски и разрешение на реализацию, эксплуатацию СМИБ.
\end{itemize}

\emph{Реализация и эксплуатация СМИБ}. Организация должна:
\begin{itemize}
\item представить и реализовать план обработки риска;
\item реализовать средства управления;
\item обеспечить повышение квалификации и информированность персонала;
\item управлять эксплуатацией, управлять ресурсами, реализовать процедуры.
\end{itemize}

\emph{Мониторинг и анализ СМИБ}. Организация должна:
\begin{itemize}
\item выполнять процедуры мониторинга;
\item проводить регулярные анализы;
\item анализировать уровень остаточного риска;
\item осуществлять внутренний аудит СМИБ;
\item предпринять анализ системы руководством;
\item регистрировать действия и события, которые могли бы повлиять на рабочие характеристики и эффективность СМИБ.
\end{itemize}

\emph{Поддержание и совершенствование СМИБ}. Организация должна:
\begin{itemize}
\item реализовывать любые сформулированные усовершенствования СМИБ;
\item предпринимать любые сформулированные исправляющие и профилактические действия;
\item сообщать их результаты всем заинтересованным сторонам;
\item проверять соответствие этих усовершенствований поставленным целям и задачам.
\end{itemize}
\subsection{Система документации, ответственность руководства, менеджмент ресурсов, обучение осознанию компетенции}
\label{sec-4_3}


Организация должна иметь систему документации для СМИБ. В этой системе документы должны соответствующим образом защищаться и проверяться. Эта система должна также охватывать любые записи, которые создаются или сохраняются для подтверждения доказательства эффективной работы СМИБ.

Эти требования к документации детальнее определены в ГОСТ Р ИСО/МЭК 27001 и ГОСТ Р ИСО 9001.

Руководство должно представить обоснование его обязательств по созданию, реализации, эксплуатации, контролю, анализу, поддержанию и совершенствованию СМИБ.

Организация должна определять и предоставлять ресурсы, необходимые для создания, реализации, эксплуатации и поддержания СМИБ; гарантирование, что процедуры политики ИБ поддерживают требования бизнеса, определения правовых и нормативных требований и обязательств по контрактам, а также обращения к ним; поддержки достаточной безопасности путём правильного применения всех реализованных средств управления, при необходимости для проведения анализа и соответствующей реакции на результаты этих анализов; при необходимости улучшение эффективности СМИБ.

Организации следует обеспечить, чтобы весь персонал, которому назначены обязанности, определённые в СМИБ, являлся компетентным для выполнения требуемых задач.

Организации также следует обеспечить, чтобы весь соответствующий персонал осознавал необходимость и важность своих действий по обеспечению ИБ, а также то, как они могут содействовать обеспечению целей СМИБ.
\subsection{Анализ СМИБ, осуществляемый руководством. Внутренний аудит, усовершенствование СМИБ}
\label{sec-4_4}


Руководство через запланированные периоды времени должно анализировать СМИБ организации, чтобы постоянно гарантировать её соответствие, достаточность и эффективность.

Подробнее эти требования изложены в ГОСТ Р ИСО/МЭУ 27001. Входные данные для анализа руководством должны содержать информацию о:
\begin{itemize}
\item результатах аудиторских проверок и анализа СМИБ;
\item ответных реакциях заинтересованных сторон;
\item методах, изделиях или процедурах, предназначенных для улучшения рабочих характеристик и эффективности СМИБ;
\item статусе профилактических и исправляющих действий;
\item уязвимости и угрозах недостаточно учтённых при предыдущем определении риска;
\item мероприятиях, проведённых по результатам предыдущих анализов;
\item любых изменениях, которые могли бы повлиять на СМИБ;
\item рекомендациях по усовершенствованию.
\end{itemize}

Выходные данные анализа руководством должны содержать любые решения и действия, относящиеся к:
\begin{itemize}
\item повышению эффективности СМИБ;
\item изменению процедур, влияющих на ИБ, что необходимо в ответ на внутренние и внешние события, которые могут повлиять на СМИБ;
\item потребностям в ресурсах.
\end{itemize}

Организация должна через запланированные интервалы проводить внутренний аудит СМИБ, чтобы определить цели контроля, средства контроля, процессы и процедуры для её СМИБ.

Организация должна непрерывно повышать эффективность СМИБ.

Организация должна определить действия по защите от будущих несоответсвий с целью предотвращения из появления.
\section{Совокупность средств управления ИБ, ориентированных на требования для электросвязи}
\label{sec-5}


Цели управления средствами управления, перечисленными ниже основаны на содержащихся в ГОСТ Р ИСО/МЭК 17999 и ГОСТ Р ИСО/МЭК 27001 требованиях.

Они адаптированы к требованиям электросвязи.

Список средств управления в данном приложении к рекомендациям является исчерпывающим, но организации следует также рассматривать другие средства управления, перечисляемые в ГОСТ Р ИСО/МЭК 17999 и ГОСТ Р ИСО/МЭК 27001.

Цели управления и средства управления выбираются как часть процессов СМИБ, определённых выше.
\subsection{Организационные меры безопасности: организационная инфраструктура ИБ, распределение обязанностей по обеспечению ИБ}
\label{sec-5_1}


Целью создания организационной инфраструктуры ИБ является управление ИБ в пределах организации электросвязи. Структуру управления ИБ следует создавать так, чтобы она способствовала инициализации и осуществлению контроля за внедрением ИБ в организации электросвязи. Следует создавать соответствующие управляющие советы с участием высшего руководства для утверждения политики ИБ, назначать ответственных лиц в области ИБ, а также осуществлять координацию и внедрение мероприятий по управлению ИБ в организации электросвязи.

При необходимости следует предусмотреть наличие специалиста по вопросам ИБ внутри организации электросвязи, к которому могут обращаться заинтересованные сотрудники.

Следует налаживать контакты с внешними специалистами по безопасности для того, чтобы быть в курсе отраслевых тенденций, способов и методов её оценки, а также с целью адекватного реагирования на инциденты нарушения ИБ.

Следует поощрять многопрофильный подход к ИБ, например путём налаживания сотрудничества между менеджерами, пользователями, администраторами, разработчиками приложений, аудиторами и сотрудниками безопасности, а также специалистами в области страхования и управления рисками.

Ответственность за защиту отдельных средств электросвязи и за выполнение конкретных процессов, обеспечивающих безопасность должна быть чётко определена.

Линейные администраторы технического обслуживания сети являются ответственными за обеспечение безопасности каждого коммутатора электросвязи.

Администратор (менеджер) технического обслуживания сети ответственен за:
\begin{itemize}
\item обеспечение того, чтобы пользовательские терминалы системы технического обслуживания сети были расположены в закрытой зоне, как описано в политике и процедурах обеспечения физической безопасности;
\item обеспечение того, чтобы регистрации пользователей коммутируемого доступа и регистрации идентификаторов пользователей были соответствующим образом установлены и поддерживались;
\item обеспечение того, чтобы разрешительные коды службы коммутации центральной станции применялись соответствующим образом;
\item поддержку мер безопасности, гарантирующих, что доступ к коммутаторам электросвязи находится под контролем.
\end{itemize}
\subsection{Менеджмент активов, ответственность за активы, идентификация активов, владение активами}
\label{sec-5_2}


Целью является достижение и поддержание защиты активов электросвязи.

Каждый актив должен быть чётко идентифицирован. Должна быть проведена и поддерживаться инвентаризация всех важных активов. Организация электросвязи должна идентифицировать все активы и задокументировать важность этих активов. Должна быть проведена и поддерживаться инвентаризация важных активов, относящихся к каждой организации электросвязи.

Существует много видов активов, относящихся к организации электросвязи, в т. ч.:
\begin{enumerate}
\item \emph{Средства коммутации}: коммутаторы для телефонной связи, интернета и подвижной связи, которые управляют информацией маршрутизации, информацией об абонентах, информацией ``чёрных списков'', зарегистрированной служебной информацией и т. п.
\item \emph{Средства передачи}: передающие ретрансляционные системы, сетевые кабели.
\item \emph{Эксплуатационные средства}: системы управления электросвязи для эксплуатации средств коммутации и передачи, которые содержат эксплуатационную информацию, информацию о повреждениях, информацию о конфигурации, информацию о клиентах, информацию о денежных расчётах, статистическую информацию о трафике и т. п.
\item \emph{Средства служб электросвязи}: информационные службы порталов, службы вызовов в кредит и по предоплате, службы через оператора, служба ADSL, ``почтовая служба'', служба построения Веб, служба подвижной связи, служба роуминга, служба подвижной ``почтовой связи'', служба вызова по номеру/справочная служба и т. п.
\item \emph{Люди}: их квалификация и способности.
\item \emph{Нематериальные средства}, такие как репутация и имидж организации.
\end{enumerate}

С целью учёта каждый актив должен иметь назначенного владельца. Термин \emph{владелец} означает лицо или сообщество, которое обладает утверждённой руководством ответственностью за управление услугами электросвязи, техническое обслуживание, использование средств электросвязи и доступ к ним. Термин \emph{владелец} не подразумевает то, что данное лицо действительно обладает каким-либо правом собственности на средство. Владение может быть распределено в соответствии с бизнес-процессом, определённой совокупностью действий, приложением/службой, определённым набором данных.
\subsection{Менеджмент активов, классификация информации: руководящие принципы классификации, маркировка и обработки информации}
\label{sec-5_3}


Целью классификации является получение информационными активами защиты соответствующего уровня. Информация и выходные данные из систем обрабатывающих классифицируемые данные должны классифицироваться с точки зрения и ценности, конфиденциальности и критичности для организации электросвязи.

Классификация и связанная с ней защитное средство контроля за информацией должные учитывать потребности бизнеса в информации совместного или ограниченного пользования и последствия для бизнеса, связанные с такими потребностями.

Указания классификации должны содержать соглашения по начальной классификации и повторной классификации через некоторое время в соответствии с некоторой предварительно определённой методикой. Классификация информационных активов может быть выполнена с точки зрения их конфиденциальности, целостности и доступности или любого другого критерия, подходящего для выражения потребностей в защите.

Информация, связанная с абонентами и клиентами должна обрабатываться с учётом её конфиденциальности. Информация, относящаяся к средствам коммутации и передачи должна управляться с учётом её критичности.

В соответствии со схемой классификации, утверждённой в организации электросвязи должна быть разработана соответствующая совокупность процедур для маркировки и обработки информации. Процедуры маркировки информации необходимы для размещения информационных активов в физическом и электронном форматах. Выходные сигналы систем, содержащих информацию, которая классифицируется как конфиденциальная или критичная должны нести соответствующую классификационную метку (на выходе).

Для каждого уровня классификации должны быть определены процедуры обработки, охватывающие безопасную обработку, хранение, передачу, переклассификацию и уничтожение. Сюда должны быть также включены процедуры регистрации любого события, относящегося к безопасности. Соглашения с другими органами электросвязи, которые содержат положения о совместном использовании информации должны содержать процедуры для определения классификации такой информации и для опознавании классификационных меток от других органов электросвязи.
\subsection{Вопросы безопасности, связанные с персоналом. Информирование и нарушениях ИБ: информирование об инцидентах безопасности, инцидентах о проблемах безопасности, инцидентах о сбоях ПО, обучение на инцидентах}
\label{sec-5_4}


Целью является минимизация вреда от инцидентов и нарушения безопасности, контроль таких инцидентов и обучение на примере инцидентов. Об инцидентах безопасности должно быть сообщено как можно быстрее по каналам управления электросвязи.

Об инцидентах безопасности, вызванных различными типами угроз, такими как, вирусы, троянские кони, черви, злонамеренные коды подвижной связи должно быть немедленно сообщено соответствующим служащим и контрагентам с использованием формальной процедуры извещения. После извещения об инциденте должна быть правильно выполнена процедура ответа об инциденте.

Для минимизации вреда, наносимого устройствам и службам электросвязи в результате инцидента должны быть выполнены ответные процессы восстановления.

При необходимости следует также сразу известить об инциденте соответствующих клиентов по прямой электронной почте и/или домашней странице, предоставляемой организацией электросвязи.

Пользователям информационных служб должна быть направлена просьба обращать внимание и сообщать о любых замеченных подозрительных уязвимостях безопасности или об угрозах системам или службам.

Организация электросвязи должна хорошо знать спецификацию системы и организацию с точки зрения безопасности и должна проявлять заботу о слабых местах и/или уязвимости системы безопасности.

Если обнаружено слабое место, то о нём следует сообщить соответствующему руководству для поддержания системы в безопасном состоянии.

Должны выполняться процедуры сообщения об отказе ПО. Должны выполняться процедуры извещения об отказе программного обеспечения в системе электросвязи.

Следует предусмотреть следующие действия:
\begin{itemize}
\item следует обращать внимание на признаки проблемы и любые сообщения, появляющиеся в системе управления электросвязи.
\item систему электросвязи, если это возможно, следует изолировать, а её использование следует прекратить.
\item следует проинформировать немедленно соответствующее контактное лицо. Если систему необходимо проверить, то её следует отсоединить от любой работающей сети электросвязи, прежде чем запустить.
\item о событии следует немедленно сообщить менеджеру по ИБ. Восстановление следует выполнять соответствующему обученному и опытному персоналу.
\end{itemize}

Должны иметься механизмы, способные оценивать и контролировать типы, количество и стоимость инцидентов и нарушений. Эту информацию следует использовать для опознавания повторных или сильно влияющих инцидентов или нарушений.
\subsection{Физическая безопасность и защита от окружающей среды}
\label{sec-5_5}
\subsubsection{Зоны безопасности: периметр физической безопасности, физические средства управления доступом.}
\label{sec-5_5_1}


Целью является предотвращение несанкционированного физического доступа, ущерба и воздействий в отношении помещений и безопасности информации организации.

Объект электросвязи должен использовать периметры безопасности (такие ограждения как стены, контролируемые с помощью карточек проходные или посты с человеком) для защиты зон, в которых содержаться устройства коммутации, передачи эксплуатации и обработки информации.

Физическая защита может быть достигнута созданием нескольких физических барьеров (преград) вокруг помещений компании и средств обработки информации. Барьеры устанавливают отдельные периметры безопасности, каждый из которых обеспечивает усиление защиты в целом. Организациям следует использовать периметры безопасности для защиты зон расположения средств обработки информации. Периметр безопасности --- это граница, создающая барьер, например, проходная, оборудованная средствами контроля входа (въезда) по идентификационным карточкам или сотрудник на стойке регистрации. Расположение и уровень защиты (стойкости) каждого барьера зависят от результатов оценки рисков.

Рекомендуется рассматривать и внедрять при необходимости следующие мероприятия по обеспечению информационной безопасности:
\begin{itemize}
\item периметр безопасности должен быть четко определен;
\item периметр здания или помещений, где расположены средства обработки информации, должен быть физически сплошным (то есть не должно быть никаких промежутков в периметре или мест, через которые можно было бы легко проникнуть). Внешние стены помещений должны иметь достаточно прочную конструкцию, а все внешние двери должны быть соответствующим образом защищены от неавторизованного доступа, например, оснащены устройствами контроля доступа, шлагбаумами, сигнализацией, замками и т.п.;
\item должна быть выделенная и укомплектованная персоналом зона регистрации посетителей или должны существовать другие мероприятия по управлению физическим доступом в помещения или здания. Доступ в помещения и здания должен быть предоставлен только авторизованному персоналу;
\item физические барьеры, в случае необходимости, должны быть расширены от пола до потолка, для предотвращения неавторизованных проникновений, а также исключения загрязнения окружающей среды в случае пожара или затоплений;
\item все противопожарные выходы в периметре безопасности должны быть оборудованы аварийной сигнализацией и плотно закрываться.
\end{itemize}

Зоны информационной безопасности необходимо защищать с помощью соответствующих мер контроля входа для обеспечения уверенности в том, что доступ позволен только авторизованному персоналу. Необходимо рассматривать следующие меры контроля:
\begin{itemize}
\item посетители зон безопасности должны сопровождаться или обладать соответствующим допуском; дату и время входа и выхода следует регистрировать. Доступ следует предоставлять только для выполнения определенных авторизованных задач. Необходимо также знакомить посетителей с требованиями безопасности и действиями на случай аварийных ситуаций;
\item доступ к важной информации и средствам ее обработки должен контролироваться и предоставляться только авторизованным лицам. Следует использовать средства аутентификации, например, карты доступа плюс PIN-код для авторизации и предоставления соответствующего доступа. Необходимо также надежным образом проводить аудит журналов регистрации доступа;
\item необходимо требовать, чтобы весь персонал носил признаки видимой идентификации, следует поощрять его внимание к незнакомым несопровождаемым посетителям, не имеющим идентификационных карт сотрудников;
\item права доступа сотрудников в зоны информационной безопасности следует регулярно анализировать и пересматривать.
\end{itemize}
\subsubsection{Защита зданий, производственных помещений и оборудования}
\label{sec-5_5_2}


Зона информационной безопасности может быть защищена путем закрытия на замок самого офиса или нескольких помещений внутри физического периметра безопасности, которые могут быть заперты и иметь запираемые файл-кабинеты или сейфы. При выборе и проектировании безопасной зоны следует принимать во внимание возможные последствия от пожара, наводнения, взрыва, уличных беспорядков и других форм природного или искусственного бедствия. Также следует принимать в расчет соответствующие правила и стандарты в отношении охраны здоровья и безопасности труда. Необходимо рассматривать также любые угрозы безопасности от соседних помещений, например затоплений.

При этом следует предусматривать следующие меры:
\begin{itemize}
\item основное оборудование должно быть расположено в местах с ограничением доступа посторонних лиц;
\item здания не должны выделяться на общем фоне и должны иметь минимальные признаки своего назначения --- не должны иметь очевидных вывесок вне или внутри здания, по которым можно сделать вывод о выполняемых функциях обработки информации;
\item подразделения поддержки и оборудование, например, фотокопировальные устройства и факсы, должны быть расположены соответствующим образом в пределах зоны безопасности во избежание доступа, который мог бы скомпрометировать информацию;
\item двери и окна необходимо запирать, когда в помещениях нет сотрудников, а также следует предусмотреть внешнюю защиту окон --- особенно, низко расположенных;
\item необходимо также внедрять соответствующие системы обнаружения вторжений для внешних дверей и доступных для этого окон, которые должны быть профессионально установлены и регулярно тестироваться. Свободные помещения необходимо ставить на сигнализацию. Аналогично следует оборудовать другие помещения, в которых расположены средства коммуникаций;
\item необходимо физически изолировать средства обработки информации, контролируемые организацией и используемые третьей стороной;
\item справочники и внутренние телефонные книги, идентифицирующие местоположения средств обработки важной информации, не должны быть доступны посторонним лицам;
\item следует обеспечивать надежное хранение опасных или горючих материалов на достаточном расстоянии от зоны информационной безопасности. Большие запасы бумаги для печатающих устройств не следует хранить в зоне безопасности без соответствующих мер пожарной безопасности;
\item резервное оборудование и носители данных следует располагать на безопасном расстоянии во избежание повреждения от последствий стихийного бедствия в основном здании.
\end{itemize}
\subsubsection{Работа в охраняемых зонах, изолированные зоны приемки и отгрузки материальных ценностей}
\label{sec-5_5_3}


Для повышения степени защиты зон информационной безопасности могут потребоваться дополнительные меры по управлению информационной безопасностью и соответствующие руководства. Они должны включать мероприятия в отношении персонала или представителей третьих сторон, работающих в зоне безопасности и состоять в следующем:
\begin{itemize}
\item о существовании зоны информационной безопасности и проводимых в ней работах должны быть осведомлены только лица, которым это необходимо в силу производственной необходимости;
\item из соображений безопасности и предотвращения возможности злонамеренных действий в охраняемых зонах необходимо избегать случаев работы без надлежащего контроля со стороны уполномоченного персонала;
\item пустующие зоны безопасности должны быть физически закрыты, и их состояние необходимо периодически проверять;
\item персоналу третьих сторон ограниченный авторизованный и контролируемый доступ в зоны безо пасности или к средствам обработки важной информации следует предоставлять только на время такой необходимости. Между зонами с различными уровнями безопасности внутри периметра безопасности могут потребоваться дополнительные барьеры и периметры ограничения физического доступа;
\item использование фото, видео, аудио или другого записывающего оборудования должно быть разрешено только при получении специального разрешения.
\end{itemize}

Зоны приемки и отгрузки материальных ценностей должны находиться под контролем и, по возможности, быть изолированы от средств обработки информации во избежание неавторизованного доступа. Требования безопасности для таких зон должны быть определены на основе оценки рисков. В этих случаях рекомендуется предусматривать следующие мероприятия:
\begin{itemize}
\item доступ к зоне складирования с внешней стороны здания должен быть разрешен только определенному и авторизованному персоналу;
\item зона складирования должна быть организована так, чтобы поступающие материальные ценности могли быть разгружены без предоставления персоналу поставщика доступа к другим частям здания;
\item должна быть обеспечена безопасность внешней(их) двери(ей) помещения для складирования, когда внутренняя дверь открыта;
\item поступающие материальные ценности должны быть осмотрены на предмет потенциальных опасностей прежде, чем они будут перемещены из помещения для складирования к местам использования;
\item поступающие материальные ценности должны быть зарегистрированы, если это необходимо.
\end{itemize}
\subsubsection{Безопасность оборудования: расположения и защита оборудования, безопасность вспомогательного оборудования}
\label{sec-5_5_4}


Цель: предотвратить потери, повреждения или компрометацию средств, а также прерывание деловой деятельности.

Оборудование должно размещено или защищено для уменьшения рисков окружающих угроз и возможностей НСД. Для защиты оборудования должны учитываться следующие руководящие указания.

Оборудование должно быть размещено так, чтобы минимизировать ненужный доступ рабочей зоны. Средства электросвязи, обрабатывающие конфиденциальные данные должны располагаться так, чтобы уменьшить риск того, что информация будет видна неуполномоченным лицам во время её использования.

Устройства, требующие специальной защиты должны быть изолированы, чтобы понизить общий уровень необходимой защиты. Должны быть утверждены средства управления для минимизации риска потенциальных физичиских угроз, например краж, пожаров, взрывов, задымления, воды (или аварии водоснабжения), пыли, вибраций, химических воздействий, помех по электропитанию, помех по линиям связи, э/м излучения, вандализма.

В частности, оборудование электросвязи должно быть выполнено и устойчиво размещено для обеспечения стойкости при землетрясениях. Для защиты от грозы должен устанавливаться грозозащитный трансформатор.

Организация должна предусмотреть указания, касающиеся возможности еды, питья и курения вблизи средств электросвязи.

Окружающие условия должны контролироваться в тех случаях, когда они могут неблагоприятно повлиять на работу средств электросвязи.

Защита от молний должна быть осуществлена во всех зданиях, а молниезащитные фильтры должны быть установлены на всех входящих силовых линиях и линиях связи.

Должно учитываться влияние любого бедствия, возникающего в соседних помещения, например пожара в соседнем здании, протечки воды с крыши или на этажах, расположенных ниже уровня земли или взрывы на улице.

Оборудование должно быть защищено от аварий электроснабжения и других нарушения, вызванных вспомогательными средствами.

Все вспомогательные средства, такие как электроснабжение, водоснабжение, канализация, отопление/вентилляция и кондиционирование воздуха должны регулярно осматриваться и, если нужно, тестироваться, чтобы обеспечить их правильное функционирование и уменьшить любой риск, вызванный из неправильной работой или отказом.

В отношении вспомогательных системы должны учитываться следующие руководящие указания.

Для оборудования, поддерживающего критические деловые операции, рекомендуется с целью обеспечения непрерывной работы или плановых прекращений работы использовать испочники песперебойного электропитания. Планы для аварийных ситуаций с электроснабжением должны предусматривать действий на случай отказа ИБП.

Оборудование ИБП должно регулярно проверяться, чтобы гарантировать, что имеет достаточную мощность и соответствует требованиям электросвязи, особенно для центров управления.

Если обработки должны продолжаться в случае продолжительного перерыва в электроснабжении, то должен быть предусмотрен резервный генереатор. Если генератор установлен, он должен регулярно проверяться на соответствие инструкциям по электросвязи.

Должны иметься достаточные запасы топлива, чтобы обеспечить работу генератора в течении продолжительного периода времени.

Аварийные выключатели электропитания должны располагаться вблизи аварийных выходов помещений, в которых размещено оборудование, чтобы облегчить быстрое выключение энергии при аварийной ситуации. На случай отказа основного источника питания должно быть обеспечено аварийное освещение. В частности, электрические устойства в изолированных зонах, таких как бызовые станции подвижной связи должны иметь возможность обеспечивать мощность, достаточную для всех нагрузок. Если это невозможно, то на уязвимом месте должен быть установлен следящий механизм для зарядной ёмкости. Для защиты от отказов электропитания должна устанавливаться аккумуляторная батарея.

В изолированной зоне мощность батареи должна быть особенно повышенной, либо должен быть установлен собственный электрические генератор достаточной мощности.
\subsubsection{Безопасность кабельной системы, обслуживание оборудования}
\label{sec-5_5_5}


Для обеспечения безопасности кабельных системы должны учитываться следующие руководящие указания.

Силовые линии и линии электросвязи, идущие к средствам обработки информации должны находиться под землёй или под полом, где это возможно, либо иметь достаточную другую защиту.

Системы сетевых кабелей должна быть защищена от несанкционированного перехвата или повреждения, например путём использования общего канала или путём обхода зон общего пользования.

Силовые кабели во избежание помех должны быть отделены от кабелей связи.

Для конфиденциальных или критических систем рассматриваются дополнительные средства управления:
\begin{enumerate}
\item Установка бронированного кабеля-провода.
\item Использование разных трасс или передающих сред, обеспечивающих соответствующую безопасность.
\item Использование волоконно-оптических кабелей.
\item Использование э/м экранирования для защиты кабелей.
\item Организация технического зондирования и физических инспекций для обнужения несанкционированных устройств, подключенных к кабелям.
\end{enumerate}

Чтобы гарантировать готовность и целостность, оборудование должно быть правильно технически облуживаться. При техническом обслуживании оборудования необходимо следующее.

Оборудование должно технически обслуживаться в соответствии со служебными интервалами и спецификациями, рекомендованными поставщиком. Выполнять ремонт и облуживать должен только уполномоченный персонал технического облуживания.

Записи должны содержать все потенциальные и действительные отказы и все случаи профилактического и восстановительного технического обслуживания.

Должны быть предусмотрены соответствующие средства контроля для обслуживания передающего оборудования, расположенного вне помещений, особенно в части удалённых, стёртых и изменённых данных. Должны выполняться все требования, налагаемые страховыми полисами.
\subsubsection{Обеспечение безопасности оборудования, используемого вне помещений организации. Безопасная утилизация или повторное использование оборудования}
\label{sec-5_5_6}


Безопасность, обеспечиваемая для оборудования, расположенного вне помещения должны быть эквивалентной той, которой обладает оборудование в помещении и используемое для той же цели, с учётом рисков работы вне помещений электросвязи.

Невзирая на право собственности, использование для обработки информации какого-либо оборудования, расположенного вне помещения электросвязи должно быть санкционированно руководством.

Должны учитываться следующие руководящие указания по защите оборудования вне помещений.

Оборудование носителей информации, находящееся вне помещений, не должны оставаться без присмотра в местах общего пользования. Портативные компьютеры должны переноситься как ручная кладь и маскироваться, где это возможно.

Инструкции производителя по защите оборудования, например по защите от сильных э/м полей должны соблюдаться постоянно. Средства контроля надомной работы должны быть установлены с помощью определения риска и подходящих средств управления, применяемых соответствующим образом, например запирающиеся картотеки, правило ``чистого стола'', контроль доступа к компьютерам и безопасная связь с офисом.

Для защиты оборудования, расположенного вне помещения должна иметься достаточная страховка. Риски безопасности, например от повреждения, кражи и подслушивания в разных местах могут значительно отличаться, что должно учитываться при определенни наиболее подходящих средств управления.

Все виды оборудования, содержащие носители информации должны быть проверены, чтобы гарантировать, что любые конфиденциальные данные и лицензионное ПО удалено или затёрто до передачи.

Устройства, содержащие конфиденциальную информацию должны быть физически разрушены, либо их информация должны быть разрушена, удалена или затёрта с использованием утверждённых методов, которые предпочтительные стандартных функций удаления или форматирования.
\subsubsection{Общие требования: правило "чистого стола" и "чистого экрана", вынос имущества}
\label{sec-5_5_7}


Цель общих требований --- предотвратить компрометацию или кражу информации и средств обработки информации. Объект электросвязи должен учитывать применение правил чистого стола по отношению к бумагам и переносимым накопителям, а к средства обработки информации - правило чистого экрана.

Правило чистого стола, чистого экрана уменьшают риск НСД, потерь и повреждений информации в обычное рабочее время и вне его. Информация, оставленная на столе может быть повреждена и разрушена также в результате бедствия, такого как пожар, землетрясения, наводнения, взрыв.

Должны учитываться следующие руководящие указания.

Бумаги и компьютерные носители, когда они не используются, особенно в нерабочие часы, должны храниться в подходящих запираемых шкафах и/или в безопасной мебели другого типа. Конфиденциальная или критичная деловая информация, когда она не требуется, в особенности когда в помещении, в которых расположы средства не заняты, должны быть заперта (идеально в сейф, либо в шкаф, защищающий от физического ущерба).

ПК, компьютерные терминалы и принтеры, не участвующие в работе, не следует оставлять зарегистрированными в системе, когда они не используются. Они должны быть защищены с помощью закрываемых на ключ замков, паролей и других средств контроля.

Пункты приёма и выдачи электронных почтовых сообщений и необслуживаемые факсы должны быть защищены. Должно быть предотвращено несанкционированное использование фотокопировальных устройств.

Конфиденцильная или классифицированная информация после печати должна немедленно удаляться из печатающих устройств.

Оборудование, информация или ПО не должны выноситься из объекта без разрешения. Должны учитываться следующие руководящие указания.
\begin{enumerate}
\item Имущество не слудует выносить из объекта без разрешения.
\item Должны быть чётко определены лица, выдающие право выноса имущества из объекта.
\item Когда необходимо и уместно, оборудование должно быть снято с учёта и вновь зарегистрировано по возврещении.
\item Для обнаружения несанкционированного выноса имущества должны быть производиться внезапные проверки.
\item Люди должны знать, что практикуются внезапные проверки.
\end{enumerate}

\end{document}
